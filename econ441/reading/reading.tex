\documentclass[12pt]{article}

% Packages for mathematical notation and symbols
\usepackage{amsmath, amssymb, amsthm}

% Page layout and margins
\usepackage[a4paper, margin=1in]{geometry}

% Font settings
\usepackage{newpxtext,newpxmath}
\usepackage[T1]{fontenc}

% Title and author
\title{Econ 441 Reading Notes}
\author{Franklin She}
\date{Spring 2024}

% Theorem-like environments
\newtheorem{theorem}{Theorem}[section]
\newtheorem{corollary}[theorem]{Corollary}
\newtheorem{lemma}[theorem]{Lemma}
\newtheorem{proposition}[theorem]{Proposition}
\theoremstyle{definition}
\newtheorem{definition}[theorem]{Definition}
\newtheorem{example}[theorem]{Example}
\theoremstyle{remark}
\newtheorem*{remark}{Remark}

% Document starts here
\begin{document}

\maketitle

\section{Introduction \& Basic Tools}

\subsection{Choice under uncertainty, expected utility}

\subsubsection{Reading: van Zandt Ch. 1 \& 2}

Overview
\begin{itemize}
    \item Weak Axiom of Revealed Preference (WARP)
    \item Independence Axiom
    \item First Order Stochastic Dominance
    \item Continuity Axiom
    \item Expected Utility Maximization
    \item von Neumann-Morgenstern Theorem
\end{itemize}

\subsubsection{Chapter 1: Introduction to decision theory}

Let $X$ be the finite set of all consumption bundles. Let $A$ be a non-empty subset of $X$ of a potential feasible set. Let $C(A)$ be the elements of $A$ that the agent might choose. $C(A)$ may contain more than one item because of indifference.

\begin{definition}(Preference Maximization): Let $\succeq$ be the relation defined for a choice rule $C(\cdot)$.

    \[
        x \in C(A) \iff x \succeq y \qquad \forall y \in A
    \]

\end{definition}
\begin{definition}[Preference Relation is Rational]
    The preference relation $\succeq$ is rational if it satisfies the following two axioms:
    \begin{enumerate}
        \item Completeness: $\forall x, y \in X$, either $x \succeq y$ or $y \succeq x$ or both.
        \item Transitivity: $\forall x, y, z \in X$, if $x \succeq y$ and $y \succeq z$, then $x \succeq z$.
    \end{enumerate}
\end{definition}

\begin{proposition} The choice rule satisfies WARP if and only if satisfies preference maximization and the preference relation is rational.

\end{proposition}

\begin{definition}(Weak Axiom of Revealed Preference): Let $x, y \in X$. Let $A$ and $B$ be subsets of $X$ containing both $x$ and $y$. If $x \in C(A)$ and $y \in C(B)$, then $x \in C(B)$. In other words, if $x$ is revealed weakly preferred to $y$, then $y$ is not revealed preferred to $x$.

\end{definition}

\begin{proposition}2. If the preference relation $\succeq$ is rational, then there is a utility function $U \colon X \to \mathbb{R}$ such that $\forall x, y \in X$, 

    \[
        x \succeq y \iff U(x) \geq U(y)
    \]

    - Note: the utility representation is not unique. Consider monotonic transformations on $U$.
    - Utility functions are useful to present an example of preferences when $X$ is large. (e.g. $U(x) = \log(x_1) + \log(x_2)$)

\end{proposition}

\subsubsection{Chapter 2: Lotteries and objected expected utility}

\begin{definition}(Probability measure): $P \colon X \to [0, 1]$ is a probability measure and $P(x)$ is the probability of outcome $x$ if and only if

    1. $P(x) \geq 0$ for all $x \in X$
    2. $\sum_{x \in X} P(x) = 1$

\end{definition}

Consequentialism: The decision maker is indifferent between a compound lottery and its reduced lottery (the lottery of the outcomes of the compound lottery).

\begin{definition}(Independence Axiom) For all lotteries $P, Q, R \in \mathcal{L}$ and $\alpha \in [0, 1]$

    \[
        P \succeq Q \iff \alpha P + (1 - \alpha) R \sim \succeq Q + (1 - \alpha) R
    \]

    - We can rewrite simple lotteries as compound lotteries to invoke the independence axiom to find inconsistencies in choices. (Exercise 2.1)

\end{definition}

\begin{definition}{First Order Stochastic Dominance}
    Let $P$ and $Q$ be two lotteries with outcomes $x_1, \dots, x_n$ and $y_1, \dots, y_n$, and $P(x_i) = Q(y_i) = \alpha_i$ for all $i$. We allow that $x_i = x_j$ or $y_i = y_j$ for some $i \neq j$. $P$ weakly first order stochastically dominates $Q$ if and only if
    \[
        x_i \geq y_i \qquad \forall i \in \{1, \dots, n\}
    \]

    $P$ (strictly) first order stochastically dominates $Q$ ($P$ f.o.s.d. $Q$) if and only if
    \[
        x_i > y_i \qquad \forall i \in \{1, \dots, n\}
    \]
\end{definition}

\begin{proposition}1. Suppose that $\succeq$ satisfies the independence axiom. If $P$ weakly (resp. strictly) first order stochastically dominates $Q$, then $P \succeq Q$ (resp. $P \succ Q$).
    - Proof with the independence axiom and the compound lottery representation of simple lotteries.
    - We can use this proposition to show conclude that lotteries are preferred to others by rewriting the lotteries into simple lotteries with the same probability distribution. (p. 22)

\end{proposition}

\begin{proposition}2. Let $X = \{x_1, \dots, x_n\}$ and WLOG assume that $x_1 \succeq \dots \succeq x_n$. $P$ weakly f.o.s.d $Q$ if and only if $\forall k \in \{1, \dots, n\}$

    \[
        \sum_{i = 1}^k P(x_i) \geq \sum_{i = 1}^k Q(x_i)
    \]

    Because the probabilities sum to 1, this can be rewritten as

    \[
        \sum_{i = k}^n P(x_i) \leq \sum_{i = k}^n Q(x_i)
    \]

    In other words, we can either compare the cumulative probabilities of the worst outcomes or the best outcomes to determine first order stochastic dominance.

\end{proposition}

\begin{definition}(Continuity Axiom) If $P, Q \in \mathcal{L}$ and $P \succ Q$, then $\forall R \in \mathcal{L}$, there exists $\alpha \in (0, 1)$ such that

    \[
        P \succ (1 - \alpha) Q + \alpha R
    \]

    and there exists $\beta \in (0, 1)$ such that

    \[
        (1 - \beta) P + \beta R \succ Q.
    \]

    In words, there is nothing so good (or so bad) that it does not become insignificant if it occurs with small enough probability. This is also called the Archimedean axiom.
    - Note: this is still unintuitive for me. There is an example on page 25.

\end{definition}

\begin{definition}(Expected utility maximization) Preference $\succeq$ over lotteries $\mathcal{L}$ satisfies expected utility maximization if there is a function $u \colon Z \to \mathbb{R}$ such that $\forall P, Q \in \mathcal{L}$ and $P \succeq Q$ if and only if

    \[
        \sum_{z \in Z} u(z) P(z) \geq \sum_{z \in Z} u(z) Q(z)
    \]

    - A decision maker is an expected utility maximizer if for some function $u$, she always prefers the lottery with the highest expected utility.

\end{definition}

\begin{theorem}1 (von Neumann-Morgenstern) If $\succeq$ satisfies the independence axiom and the continuity axiom, then $\succeq$ satisfies expected utility maximization.
    - Because VNM utility functions measure strength of preferences over outcomes, they are sometimes called cardinal utility functions.
    - Only positive affine transformations preserve the ranking of lotteries. If $u \colon X \to \mathbb{R}$ is a VNM utility function, then $v \colon X \to \mathbb{R}$ is a positive affine transformation of $u$ if and only if there exists $a > 0$ and $b$ such that $v(x) = a + b u(x)$ for all $x \in X$.

\end{theorem}

\subsection{Bayesian inference}

\subsubsection{Reading: van Zandt Ch. 4}

Chapter 4: Choosing when there is new information

Conditional beliefs:
- Let $\pi \colon S \to \mathbb{R}$ (a probability measure) be the prior belief.
- Let $\pi(\cdot \mid E) \colon S \to \mathbb{R}$ be the posterior belief after observing $E$.
- The general rule is (to rescale the probabilities of possible states after observing $E$, some states may be impossible after observing $E$)

\[
    \pi(s \mid E) = \begin{cases}
        \frac{\pi(s)}{\pi(E)} & \text{if } s \in E \\
        0 & \text{if } s \notin E
    \end{cases}
\]

Bayes' Rule

\[
    \pi(A \mid E) = \frac{\pi(A \cap E)}{\pi(E)}
\]

\end{document}

