% Options for packages loaded elsewhere
\PassOptionsToPackage{unicode}{hyperref}
\PassOptionsToPackage{hyphens}{url}
%
\documentclass[
]{article}
\usepackage{amsmath,amssymb}
\usepackage{iftex}
\ifPDFTeX
  \usepackage[T1]{fontenc}
  \usepackage[utf8]{inputenc}
  \usepackage{textcomp} % provide euro and other symbols
\else % if luatex or xetex
  \usepackage{unicode-math} % this also loads fontspec
  \defaultfontfeatures{Scale=MatchLowercase}
  \defaultfontfeatures[\rmfamily]{Ligatures=TeX,Scale=1}
\fi
\usepackage{lmodern}
\ifPDFTeX\else
  % xetex/luatex font selection
\fi
% Use upquote if available, for straight quotes in verbatim environments
\IfFileExists{upquote.sty}{\usepackage{upquote}}{}
\IfFileExists{microtype.sty}{% use microtype if available
  \usepackage[]{microtype}
  \UseMicrotypeSet[protrusion]{basicmath} % disable protrusion for tt fonts
}{}
\makeatletter
\@ifundefined{KOMAClassName}{% if non-KOMA class
  \IfFileExists{parskip.sty}{%
    \usepackage{parskip}
  }{% else
    \setlength{\parindent}{0pt}
    \setlength{\parskip}{6pt plus 2pt minus 1pt}}
}{% if KOMA class
  \KOMAoptions{parskip=half}}
\makeatother
\usepackage{xcolor}
\usepackage[margin=1in]{geometry}
\usepackage{color}
\usepackage{fancyvrb}
\newcommand{\VerbBar}{|}
\newcommand{\VERB}{\Verb[commandchars=\\\{\}]}
\DefineVerbatimEnvironment{Highlighting}{Verbatim}{commandchars=\\\{\}}
% Add ',fontsize=\small' for more characters per line
\usepackage{framed}
\definecolor{shadecolor}{RGB}{248,248,248}
\newenvironment{Shaded}{\begin{snugshade}}{\end{snugshade}}
\newcommand{\AlertTok}[1]{\textcolor[rgb]{0.94,0.16,0.16}{#1}}
\newcommand{\AnnotationTok}[1]{\textcolor[rgb]{0.56,0.35,0.01}{\textbf{\textit{#1}}}}
\newcommand{\AttributeTok}[1]{\textcolor[rgb]{0.13,0.29,0.53}{#1}}
\newcommand{\BaseNTok}[1]{\textcolor[rgb]{0.00,0.00,0.81}{#1}}
\newcommand{\BuiltInTok}[1]{#1}
\newcommand{\CharTok}[1]{\textcolor[rgb]{0.31,0.60,0.02}{#1}}
\newcommand{\CommentTok}[1]{\textcolor[rgb]{0.56,0.35,0.01}{\textit{#1}}}
\newcommand{\CommentVarTok}[1]{\textcolor[rgb]{0.56,0.35,0.01}{\textbf{\textit{#1}}}}
\newcommand{\ConstantTok}[1]{\textcolor[rgb]{0.56,0.35,0.01}{#1}}
\newcommand{\ControlFlowTok}[1]{\textcolor[rgb]{0.13,0.29,0.53}{\textbf{#1}}}
\newcommand{\DataTypeTok}[1]{\textcolor[rgb]{0.13,0.29,0.53}{#1}}
\newcommand{\DecValTok}[1]{\textcolor[rgb]{0.00,0.00,0.81}{#1}}
\newcommand{\DocumentationTok}[1]{\textcolor[rgb]{0.56,0.35,0.01}{\textbf{\textit{#1}}}}
\newcommand{\ErrorTok}[1]{\textcolor[rgb]{0.64,0.00,0.00}{\textbf{#1}}}
\newcommand{\ExtensionTok}[1]{#1}
\newcommand{\FloatTok}[1]{\textcolor[rgb]{0.00,0.00,0.81}{#1}}
\newcommand{\FunctionTok}[1]{\textcolor[rgb]{0.13,0.29,0.53}{\textbf{#1}}}
\newcommand{\ImportTok}[1]{#1}
\newcommand{\InformationTok}[1]{\textcolor[rgb]{0.56,0.35,0.01}{\textbf{\textit{#1}}}}
\newcommand{\KeywordTok}[1]{\textcolor[rgb]{0.13,0.29,0.53}{\textbf{#1}}}
\newcommand{\NormalTok}[1]{#1}
\newcommand{\OperatorTok}[1]{\textcolor[rgb]{0.81,0.36,0.00}{\textbf{#1}}}
\newcommand{\OtherTok}[1]{\textcolor[rgb]{0.56,0.35,0.01}{#1}}
\newcommand{\PreprocessorTok}[1]{\textcolor[rgb]{0.56,0.35,0.01}{\textit{#1}}}
\newcommand{\RegionMarkerTok}[1]{#1}
\newcommand{\SpecialCharTok}[1]{\textcolor[rgb]{0.81,0.36,0.00}{\textbf{#1}}}
\newcommand{\SpecialStringTok}[1]{\textcolor[rgb]{0.31,0.60,0.02}{#1}}
\newcommand{\StringTok}[1]{\textcolor[rgb]{0.31,0.60,0.02}{#1}}
\newcommand{\VariableTok}[1]{\textcolor[rgb]{0.00,0.00,0.00}{#1}}
\newcommand{\VerbatimStringTok}[1]{\textcolor[rgb]{0.31,0.60,0.02}{#1}}
\newcommand{\WarningTok}[1]{\textcolor[rgb]{0.56,0.35,0.01}{\textbf{\textit{#1}}}}
\usepackage{graphicx}
\makeatletter
\def\maxwidth{\ifdim\Gin@nat@width>\linewidth\linewidth\else\Gin@nat@width\fi}
\def\maxheight{\ifdim\Gin@nat@height>\textheight\textheight\else\Gin@nat@height\fi}
\makeatother
% Scale images if necessary, so that they will not overflow the page
% margins by default, and it is still possible to overwrite the defaults
% using explicit options in \includegraphics[width, height, ...]{}
\setkeys{Gin}{width=\maxwidth,height=\maxheight,keepaspectratio}
% Set default figure placement to htbp
\makeatletter
\def\fps@figure{htbp}
\makeatother
\setlength{\emergencystretch}{3em} % prevent overfull lines
\providecommand{\tightlist}{%
  \setlength{\itemsep}{0pt}\setlength{\parskip}{0pt}}
\setcounter{secnumdepth}{-\maxdimen} % remove section numbering
\ifLuaTeX
  \usepackage{selnolig}  % disable illegal ligatures
\fi
\IfFileExists{bookmark.sty}{\usepackage{bookmark}}{\usepackage{hyperref}}
\IfFileExists{xurl.sty}{\usepackage{xurl}}{} % add URL line breaks if available
\urlstyle{same}
\hypersetup{
  pdftitle={pset2\_R},
  pdfauthor={Franklin She},
  hidelinks,
  pdfcreator={LaTeX via pandoc}}

\title{pset2\_R}
\author{Franklin She}
\date{2024-02-15}

\begin{document}
\maketitle

\hypertarget{part-ii-2-r-questions}{%
\section{Part II: 2 R Questions}\label{part-ii-2-r-questions}}

\hypertarget{question-1.-load-data}{%
\subsection{Question 1. Load Data}\label{question-1.-load-data}}

\begin{Shaded}
\begin{Highlighting}[]
\FunctionTok{library}\NormalTok{(haven)}

\NormalTok{financeR\_data }\OtherTok{\textless{}{-}} \FunctionTok{read\_dta}\NormalTok{(}\StringTok{"financeR.dta"}\NormalTok{)}
\FunctionTok{head}\NormalTok{(financeR\_data)}
\end{Highlighting}
\end{Shaded}

\begin{verbatim}
## # A tibble: 6 x 5
##   date          r_A       r_B     r_C     r_M
##   <chr>       <dbl>     <dbl>   <dbl>   <dbl>
## 1 12/1/2004  0.0191 -0.00585   0.100   0.0386
## 2 1/1/2005  -0.0508  0.0975    0.0227  0.0325
## 3 2/1/2005  -0.0207  0.00858  -0.0170 -0.0253
## 4 3/1/2005   0.0287  0.000354  0.0625  0.0189
## 5 4/1/2005  -0.0162  0.0179   -0.0195 -0.0191
## 6 5/1/2005   0.0124 -0.125     0.0127 -0.0201
\end{verbatim}

\hypertarget{question-2-number-of-variables-and-observations}{%
\subsection{Question 2: Number of Variables and
Observations}\label{question-2-number-of-variables-and-observations}}

\begin{Shaded}
\begin{Highlighting}[]
\NormalTok{data\_dimensions }\OtherTok{\textless{}{-}} \FunctionTok{dim}\NormalTok{(financeR\_data)}
\NormalTok{num\_observations }\OtherTok{\textless{}{-}}\NormalTok{ data\_dimensions[}\DecValTok{1}\NormalTok{]}
\NormalTok{num\_variables }\OtherTok{\textless{}{-}}\NormalTok{ data\_dimensions[}\DecValTok{2}\NormalTok{]}

\FunctionTok{cat}\NormalTok{(}\StringTok{"Number of observations (rows):"}\NormalTok{, num\_observations, }\StringTok{"}\SpecialCharTok{\textbackslash{}n}\StringTok{"}\NormalTok{)}
\end{Highlighting}
\end{Shaded}

\begin{verbatim}
## Number of observations (rows): 156
\end{verbatim}

\begin{Shaded}
\begin{Highlighting}[]
\FunctionTok{cat}\NormalTok{(}\StringTok{"Number of variables (columns):"}\NormalTok{, num\_variables, }\StringTok{"}\SpecialCharTok{\textbackslash{}n}\StringTok{"}\NormalTok{)}
\end{Highlighting}
\end{Shaded}

\begin{verbatim}
## Number of variables (columns): 5
\end{verbatim}

\hypertarget{question-3-data-types-of-variables}{%
\subsection{Question 3: Data Types of
Variables}\label{question-3-data-types-of-variables}}

\begin{Shaded}
\begin{Highlighting}[]
\NormalTok{variable\_classes }\OtherTok{\textless{}{-}} \FunctionTok{sapply}\NormalTok{(financeR\_data, class)}

\FunctionTok{print}\NormalTok{(variable\_classes)}
\end{Highlighting}
\end{Shaded}

\begin{verbatim}
##        date         r_A         r_B         r_C         r_M 
## "character"   "numeric"   "numeric"   "numeric"   "numeric"
\end{verbatim}

\begin{Shaded}
\begin{Highlighting}[]
\NormalTok{non\_numeric\_variables }\OtherTok{\textless{}{-}} \FunctionTok{names}\NormalTok{(variable\_classes[variable\_classes }\SpecialCharTok{!=} \StringTok{"numeric"}\NormalTok{])}

\ControlFlowTok{if}\NormalTok{(}\FunctionTok{length}\NormalTok{(non\_numeric\_variables) }\SpecialCharTok{\textgreater{}} \DecValTok{0}\NormalTok{) \{}
  \FunctionTok{cat}\NormalTok{(}\StringTok{"Non{-}numeric variable(s):"}\NormalTok{, }\FunctionTok{paste}\NormalTok{(non\_numeric\_variables, }\AttributeTok{collapse =} \StringTok{", "}\NormalTok{), }\StringTok{"}\SpecialCharTok{\textbackslash{}n}\StringTok{"}\NormalTok{)}
\NormalTok{\}}
\end{Highlighting}
\end{Shaded}

\begin{verbatim}
## Non-numeric variable(s): date
\end{verbatim}

\hypertarget{question-4-drop-non-numeric-variable}{%
\subsection{Question 4: Drop Non-numeric
Variable}\label{question-4-drop-non-numeric-variable}}

\begin{Shaded}
\begin{Highlighting}[]
\FunctionTok{library}\NormalTok{(dplyr)}
\end{Highlighting}
\end{Shaded}

\begin{verbatim}
## 
## Attaching package: 'dplyr'
\end{verbatim}

\begin{verbatim}
## The following objects are masked from 'package:stats':
## 
##     filter, lag
\end{verbatim}

\begin{verbatim}
## The following objects are masked from 'package:base':
## 
##     intersect, setdiff, setequal, union
\end{verbatim}

\begin{Shaded}
\begin{Highlighting}[]
\NormalTok{financeR\_data\_cleaned }\OtherTok{\textless{}{-}} \FunctionTok{select}\NormalTok{(financeR\_data, }\SpecialCharTok{{-}}\FunctionTok{one\_of}\NormalTok{(non\_numeric\_variables))}
\FunctionTok{head}\NormalTok{(financeR\_data\_cleaned)}
\end{Highlighting}
\end{Shaded}

\begin{verbatim}
## # A tibble: 6 x 4
##       r_A       r_B     r_C     r_M
##     <dbl>     <dbl>   <dbl>   <dbl>
## 1  0.0191 -0.00585   0.100   0.0386
## 2 -0.0508  0.0975    0.0227  0.0325
## 3 -0.0207  0.00858  -0.0170 -0.0253
## 4  0.0287  0.000354  0.0625  0.0189
## 5 -0.0162  0.0179   -0.0195 -0.0191
## 6  0.0124 -0.125     0.0127 -0.0201
\end{verbatim}

\hypertarget{question-5-calculate-excess-returns}{%
\subsection{Question 5: Calculate Excess
Returns}\label{question-5-calculate-excess-returns}}

\begin{Shaded}
\begin{Highlighting}[]
\NormalTok{rf }\OtherTok{\textless{}{-}} \FloatTok{0.0041}

\NormalTok{calculate\_excess\_returns }\OtherTok{\textless{}{-}} \ControlFlowTok{function}\NormalTok{(returns) \{}
\NormalTok{  excess\_returns }\OtherTok{\textless{}{-}}\NormalTok{ returns }\SpecialCharTok{{-}}\NormalTok{ rf}
  \FunctionTok{return}\NormalTok{(excess\_returns)}
\NormalTok{\}}

\NormalTok{financeR\_excess\_returns }\OtherTok{\textless{}{-}} \FunctionTok{as.data.frame}\NormalTok{(}\FunctionTok{sapply}\NormalTok{(financeR\_data\_cleaned, calculate\_excess\_returns))}

\FunctionTok{head}\NormalTok{(financeR\_excess\_returns)}
\end{Highlighting}
\end{Shaded}

\begin{verbatim}
##            r_A          r_B          r_C         r_M
## 1  0.015031184 -0.009947938  0.095899987  0.03445962
## 2 -0.054895037  0.093351001  0.018627288  0.02839332
## 3 -0.024807292  0.004476014 -0.021137002 -0.02939047
## 4  0.024647899 -0.003745696  0.058447088  0.01480335
## 5 -0.020266300  0.013785573 -0.023603586 -0.02321766
## 6  0.008341376 -0.129013022  0.008558242 -0.02420858
\end{verbatim}

\hypertarget{question-6-descriptive-statistics-and-highest-mean-return}{%
\subsection{Question 6: Descriptive Statistics and Highest Mean
Return}\label{question-6-descriptive-statistics-and-highest-mean-return}}

\begin{Shaded}
\begin{Highlighting}[]
\NormalTok{descriptive\_stats }\OtherTok{\textless{}{-}} \FunctionTok{summary}\NormalTok{(financeR\_excess\_returns)}

\FunctionTok{print}\NormalTok{(descriptive\_stats)}
\end{Highlighting}
\end{Shaded}

\begin{verbatim}
##       r_A                 r_B                 r_C           
##  Min.   :-0.174342   Min.   :-0.462632   Min.   :-0.713305  
##  1st Qu.:-0.029586   1st Qu.:-0.043620   1st Qu.:-0.082507  
##  Median : 0.001516   Median : 0.004074   Median :-0.005886  
##  Mean   : 0.003752   Mean   : 0.001441   Mean   : 0.007011  
##  3rd Qu.: 0.042496   3rd Qu.: 0.057011   3rd Qu.: 0.058521  
##  Max.   : 0.129233   Max.   : 0.287812   Max.   : 1.799471  
##       r_M           
##  Min.   :-0.172031  
##  1st Qu.:-0.020429  
##  Median : 0.006967  
##  Mean   : 0.001999  
##  3rd Qu.: 0.025580  
##  Max.   : 0.101795
\end{verbatim}

\begin{Shaded}
\begin{Highlighting}[]
\NormalTok{mean\_returns }\OtherTok{\textless{}{-}} \FunctionTok{sapply}\NormalTok{(financeR\_excess\_returns, mean)}

\FunctionTok{cat}\NormalTok{(}\StringTok{"The asset with the highest mean return is:"}\NormalTok{, }\FunctionTok{names}\NormalTok{(}\FunctionTok{which.max}\NormalTok{(mean\_returns)), }\StringTok{"with a mean return of"}\NormalTok{, }\FunctionTok{max}\NormalTok{(mean\_returns), }\StringTok{"}\SpecialCharTok{\textbackslash{}n}\StringTok{"}\NormalTok{)}
\end{Highlighting}
\end{Shaded}

\begin{verbatim}
## The asset with the highest mean return is: r_C with a mean return of 0.007011021
\end{verbatim}

\hypertarget{question-7-variances-and-covariances}{%
\subsection{Question 7: Variances and
Covariances}\label{question-7-variances-and-covariances}}

\begin{Shaded}
\begin{Highlighting}[]
\NormalTok{var\_cov\_matrix }\OtherTok{\textless{}{-}} \FunctionTok{var}\NormalTok{(financeR\_excess\_returns)}
\FunctionTok{print}\NormalTok{(var\_cov\_matrix)}
\end{Highlighting}
\end{Shaded}

\begin{verbatim}
##               r_A           r_B          r_C          r_M
## r_A  2.905690e-03 -0.0003844852 0.0001249001 1.171093e-05
## r_B -3.844852e-04  0.0105434580 0.0132516055 2.646404e-03
## r_C  1.249001e-04  0.0132516055 0.0563354184 4.679460e-03
## r_M  1.171093e-05  0.0026464043 0.0046794598 1.615544e-03
\end{verbatim}

\begin{Shaded}
\begin{Highlighting}[]
\NormalTok{variances }\OtherTok{\textless{}{-}} \FunctionTok{diag}\NormalTok{(var\_cov\_matrix)}
\NormalTok{most\_volatile\_asset }\OtherTok{\textless{}{-}} \FunctionTok{names}\NormalTok{(}\FunctionTok{which.max}\NormalTok{(variances))}
\FunctionTok{cat}\NormalTok{(}\StringTok{"(a) The most volatile asset is:"}\NormalTok{, most\_volatile\_asset, }\StringTok{"with a variance of"}\NormalTok{, }\FunctionTok{max}\NormalTok{(variances), }\StringTok{"}\SpecialCharTok{\textbackslash{}n}\StringTok{"}\NormalTok{)}
\end{Highlighting}
\end{Shaded}

\begin{verbatim}
## (a) The most volatile asset is: r_C with a variance of 0.05633542
\end{verbatim}

\begin{Shaded}
\begin{Highlighting}[]
\FunctionTok{diag}\NormalTok{(var\_cov\_matrix) }\OtherTok{\textless{}{-}} \DecValTok{0}
\NormalTok{highest\_cov\_value }\OtherTok{\textless{}{-}} \FunctionTok{max}\NormalTok{(var\_cov\_matrix)}
\NormalTok{highest\_cov\_assets }\OtherTok{\textless{}{-}} \FunctionTok{which}\NormalTok{(var\_cov\_matrix }\SpecialCharTok{==}\NormalTok{ highest\_cov\_value, }\AttributeTok{arr.ind =} \ConstantTok{TRUE}\NormalTok{)}
\NormalTok{asset\_names }\OtherTok{\textless{}{-}} \FunctionTok{rownames}\NormalTok{(var\_cov\_matrix)}
\NormalTok{pair\_with\_highest\_covariance }\OtherTok{\textless{}{-}} \FunctionTok{c}\NormalTok{(asset\_names[highest\_cov\_assets[}\DecValTok{1}\NormalTok{, }\StringTok{"row"}\NormalTok{]], asset\_names[highest\_cov\_assets[}\DecValTok{1}\NormalTok{, }\StringTok{"col"}\NormalTok{]])}
\FunctionTok{cat}\NormalTok{(}\StringTok{"(b) The two assets with the highest covariance are:"}\NormalTok{, pair\_with\_highest\_covariance[}\DecValTok{1}\NormalTok{], }\StringTok{"and"}\NormalTok{, pair\_with\_highest\_covariance[}\DecValTok{2}\NormalTok{], }\StringTok{"with a covariance of"}\NormalTok{, highest\_cov\_value, }\StringTok{"}\SpecialCharTok{\textbackslash{}n}\StringTok{"}\NormalTok{)}
\end{Highlighting}
\end{Shaded}

\begin{verbatim}
## (b) The two assets with the highest covariance are: r_C and r_B with a covariance of 0.01325161
\end{verbatim}

\hypertarget{c}{%
\subsubsection{(c)}\label{c}}

Genworth Financial (GNW) is the most volatile asset likely due to its
direct exposure to the highly fluctuating financial sector. The high
covariance between Morgan Stanley (MS) and Genworth Financial (GNW)
suggests a strong positive relationship, likely because both are deeply
integrated within the financial industry, making their performances
susceptible to similar economic and sector-specific influences.

\hypertarget{question-8-risk-return-tradeoff-analysis}{%
\subsection{Question 8: Risk-Return Tradeoff
Analysis}\label{question-8-risk-return-tradeoff-analysis}}

\begin{Shaded}
\begin{Highlighting}[]
\NormalTok{asset\_means }\OtherTok{\textless{}{-}} \FunctionTok{sapply}\NormalTok{(financeR\_excess\_returns, mean)}
\NormalTok{asset\_variances }\OtherTok{\textless{}{-}} \FunctionTok{sapply}\NormalTok{(financeR\_excess\_returns, var)}
\FunctionTok{print}\NormalTok{(asset\_means)}
\end{Highlighting}
\end{Shaded}

\begin{verbatim}
##         r_A         r_B         r_C         r_M 
## 0.003752189 0.001440923 0.007011021 0.001999263
\end{verbatim}

\begin{Shaded}
\begin{Highlighting}[]
\FunctionTok{print}\NormalTok{(asset\_variances)}
\end{Highlighting}
\end{Shaded}

\begin{verbatim}
##         r_A         r_B         r_C         r_M 
## 0.002905690 0.010543458 0.056335418 0.001615544
\end{verbatim}

\begin{Shaded}
\begin{Highlighting}[]
\FunctionTok{library}\NormalTok{(ggplot2)}

\NormalTok{risk\_return\_df }\OtherTok{\textless{}{-}} \FunctionTok{data.frame}\NormalTok{(}\AttributeTok{Asset =} \FunctionTok{names}\NormalTok{(asset\_means), }\AttributeTok{Er =}\NormalTok{ asset\_means, }\AttributeTok{Var =}\NormalTok{ asset\_variances)}

\FunctionTok{ggplot}\NormalTok{(risk\_return\_df, }\FunctionTok{aes}\NormalTok{(}\AttributeTok{x =}\NormalTok{ Var, }\AttributeTok{y =}\NormalTok{ Er, }\AttributeTok{color =}\NormalTok{ Asset)) }\SpecialCharTok{+}
  \FunctionTok{geom\_point}\NormalTok{(}\AttributeTok{size =} \DecValTok{5}\NormalTok{) }\SpecialCharTok{+} \FunctionTok{theme\_bw}\NormalTok{(}\AttributeTok{base\_size =} \DecValTok{14}\NormalTok{) }\SpecialCharTok{+}
  \FunctionTok{ggtitle}\NormalTok{(}\StringTok{"Risk{-}Return Tradeoff"}\NormalTok{) }\SpecialCharTok{+}
  \FunctionTok{xlab}\NormalTok{(}\StringTok{"Variance (Risk)"}\NormalTok{) }\SpecialCharTok{+} \FunctionTok{ylab}\NormalTok{(}\StringTok{"Expected Returns"}\NormalTok{)}
\end{Highlighting}
\end{Shaded}

\includegraphics{pset2_R_files/figure-latex/sample-means-1.pdf}

\hypertarget{d}{%
\subsubsection{(d)}\label{d}}

The risk-return tradeoff displayed in the figure indicates that Genworth
Financial (r\_C) offers the highest expected return but also comes with
the highest level of risk (variance), suggesting a potential reward for
investors willing to accept greater volatility. Conversely, the S\&P 500
index (r\_M) demonstrates the lowest risk and offers lower expected
returns, which is typical for a diversified market index. The tradeoff
between risk and expected return is a fundamental concept in finance,
where higher risk is typically associated with the potential for higher
returns, as investors demand compensation for bearing additional risk.

\hypertarget{question-9-sharpe-ratio-calculation}{%
\subsection{Question 9: Sharpe Ratio
Calculation}\label{question-9-sharpe-ratio-calculation}}

\begin{Shaded}
\begin{Highlighting}[]
\NormalTok{calculate\_sharpe\_ratio }\OtherTok{\textless{}{-}} \ControlFlowTok{function}\NormalTok{(return, variance, rf) \{}
\NormalTok{  sharpe\_ratio }\OtherTok{\textless{}{-}}\NormalTok{ (return }\SpecialCharTok{{-}}\NormalTok{ rf) }\SpecialCharTok{/} \FunctionTok{sqrt}\NormalTok{(variance)}
  \FunctionTok{return}\NormalTok{(sharpe\_ratio)}
\NormalTok{\}}

\NormalTok{sharpe\_ratios }\OtherTok{\textless{}{-}} \FunctionTok{sapply}\NormalTok{(}\DecValTok{1}\SpecialCharTok{:}\FunctionTok{ncol}\NormalTok{(financeR\_excess\_returns), }\ControlFlowTok{function}\NormalTok{(i) \{}
  \FunctionTok{calculate\_sharpe\_ratio}\NormalTok{(asset\_means[i], asset\_variances[i], rf)}
\NormalTok{\})}

\FunctionTok{names}\NormalTok{(sharpe\_ratios) }\OtherTok{\textless{}{-}} \FunctionTok{names}\NormalTok{(asset\_means)}
\FunctionTok{print}\NormalTok{(sharpe\_ratios)}
\end{Highlighting}
\end{Shaded}

\begin{verbatim}
##          r_A          r_B          r_C          r_M 
## -0.006452367 -0.025896396  0.012264635 -0.052265164
\end{verbatim}

\begin{Shaded}
\begin{Highlighting}[]
\FunctionTok{cat}\NormalTok{(}\StringTok{"The asset with the best estimated risk{-}reward tradeoff according to the Sharpe ratio is:"}\NormalTok{, }\FunctionTok{names}\NormalTok{(}\FunctionTok{which.max}\NormalTok{(sharpe\_ratios)), }\StringTok{"}\SpecialCharTok{\textbackslash{}n}\StringTok{"}\NormalTok{)}
\end{Highlighting}
\end{Shaded}

\begin{verbatim}
## The asset with the best estimated risk-reward tradeoff according to the Sharpe ratio is: r_C
\end{verbatim}

\hypertarget{question-10-portfolio-diversification}{%
\subsection{Question 10: Portfolio
Diversification}\label{question-10-portfolio-diversification}}

\hypertarget{a}{%
\subsubsection{(a)}\label{a}}

Diversification can reduce the portfolio's overall risk if the returns
on GLD and the S\&P 500 are not perfectly correlated. The rule for the
variance of a sum (i.e., the portfolio variance) indicates that if two
assets do not move exactly in tandem (less than perfect correlation),
combining them can lead to a portfolio variance that is lower than the
weighted average of the individual variances, thereby reducing risk.

\begin{Shaded}
\begin{Highlighting}[]
\NormalTok{omega }\OtherTok{\textless{}{-}} \FunctionTok{seq}\NormalTok{(}\AttributeTok{from =} \DecValTok{0}\NormalTok{, }\AttributeTok{to =} \DecValTok{1}\NormalTok{, }\AttributeTok{length.out =} \DecValTok{1000}\NormalTok{)}

\NormalTok{mean\_GLD }\OtherTok{\textless{}{-}}\NormalTok{ asset\_means[}\StringTok{\textquotesingle{}r\_A\textquotesingle{}}\NormalTok{]}
\NormalTok{mean\_SP500 }\OtherTok{\textless{}{-}}\NormalTok{ asset\_means[}\StringTok{\textquotesingle{}r\_M\textquotesingle{}}\NormalTok{]}
\NormalTok{var\_GLD }\OtherTok{\textless{}{-}}\NormalTok{ asset\_variances[}\StringTok{\textquotesingle{}r\_A\textquotesingle{}}\NormalTok{]}
\NormalTok{var\_SP500 }\OtherTok{\textless{}{-}}\NormalTok{ asset\_variances[}\StringTok{\textquotesingle{}r\_M\textquotesingle{}}\NormalTok{]}
\NormalTok{cov\_GLD\_SP500 }\OtherTok{\textless{}{-}}\NormalTok{ var\_cov\_matrix[}\StringTok{\textquotesingle{}r\_A\textquotesingle{}}\NormalTok{, }\StringTok{\textquotesingle{}r\_M\textquotesingle{}}\NormalTok{]}

\NormalTok{portfolio\_df }\OtherTok{\textless{}{-}} \FunctionTok{data.frame}\NormalTok{(}\AttributeTok{omega =}\NormalTok{ omega)}

\CommentTok{\# Calculate expected return and variance for each weight combination}
\NormalTok{portfolio\_df}\SpecialCharTok{$}\NormalTok{expected\_return }\OtherTok{\textless{}{-}}\NormalTok{ (omega }\SpecialCharTok{*}\NormalTok{ mean\_GLD) }\SpecialCharTok{+}\NormalTok{ ((}\DecValTok{1} \SpecialCharTok{{-}}\NormalTok{ omega) }\SpecialCharTok{*}\NormalTok{ mean\_SP500)}
\NormalTok{portfolio\_df}\SpecialCharTok{$}\NormalTok{portfolio\_variance }\OtherTok{\textless{}{-}}\NormalTok{ (omega}\SpecialCharTok{\^{}}\DecValTok{2} \SpecialCharTok{*}\NormalTok{ var\_GLD) }\SpecialCharTok{+}\NormalTok{ ((}\DecValTok{1} \SpecialCharTok{{-}}\NormalTok{ omega)}\SpecialCharTok{\^{}}\DecValTok{2} \SpecialCharTok{*}\NormalTok{ var\_SP500) }\SpecialCharTok{+}\NormalTok{ (}\DecValTok{2} \SpecialCharTok{*}\NormalTok{ omega }\SpecialCharTok{*}\NormalTok{ (}\DecValTok{1} \SpecialCharTok{{-}}\NormalTok{ omega) }\SpecialCharTok{*}\NormalTok{ cov\_GLD\_SP500)}

\FunctionTok{ggplot}\NormalTok{(portfolio\_df, }\FunctionTok{aes}\NormalTok{(}\AttributeTok{x =}\NormalTok{ portfolio\_variance, }\AttributeTok{y =}\NormalTok{ expected\_return, }\AttributeTok{color =}\NormalTok{ omega)) }\SpecialCharTok{+}
  \FunctionTok{geom\_point}\NormalTok{() }\SpecialCharTok{+}
  \FunctionTok{theme\_bw}\NormalTok{() }\SpecialCharTok{+}
  \FunctionTok{labs}\NormalTok{(}\AttributeTok{title =} \StringTok{"Expected Return and Volatility for Different Portfolio Weights"}\NormalTok{,}
       \AttributeTok{x =} \StringTok{"Volatility (Portfolio Variance)"}\NormalTok{,}
       \AttributeTok{y =} \StringTok{"Expected Return"}\NormalTok{,}
       \AttributeTok{color =} \StringTok{"Portfolio Weight"}\NormalTok{)}
\end{Highlighting}
\end{Shaded}

\includegraphics{pset2_R_files/figure-latex/diversification-1.pdf}

\hypertarget{c-1}{%
\subsubsection{(c)}\label{c-1}}

The risk-return tradeoff for the portfolio versus holding only SPDR Gold
or only the S\&P 500 index can be evaluated by comparing their Sharpe
ratios, expected returns, and volatilities. If the combined portfolio
offers a higher Sharpe ratio than either of the individual assets, it
suggests a better risk-adjusted return, making it a more attractive
choice for investors.

Given the estimates, an investor would never invest 100\% in the S\&P
500 index as putting at that position, putting more portfolio weight on
GLD gives a higher expected return while lowering volatility.

\end{document}
