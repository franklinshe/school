\documentclass[10pt]{article}
 
\usepackage[margin=1in]{geometry} 
\usepackage{amsmath,amsthm,amssymb, graphicx, multicol, array, bbm}
\usepackage{listings}
\usepackage{color}

\lstdefinestyle{Rstyle}{
  language=R,
  basicstyle=\small\ttfamily,
  keywordstyle=\color{blue},
  commentstyle=\color{black},
  stringstyle=\color{purple},
  numbers=left,
  numberstyle=\tiny,
  stepnumber=1,
  numbersep=5pt,
  backgroundcolor=\color{white},
  frame=single,
  rulecolor=\color{black},
  captionpos=b,
  breaklines=true,
  breakatwhitespace=true,
  title=\lstname,
  showstringspaces=false,
  tabsize=2,
  belowskip=1em,
  aboveskip=1em,
}

 
\newcommand{\N}{\mathbb{N}}
\newcommand{\Z}{\mathbb{Z}}
\newcommand{\R}{\mathbb{R}}
\newcommand{\C}{\mathbb{C}}
\newcommand{\A}{\mathcal{A}}
\newcommand{\E}{\mathbb{E}}
 
\newenvironment{problem}[2][Problem]{\begin{trivlist}
\item[\hskip \labelsep {\bfseries #1}\hskip \labelsep {\bfseries #2.}]}{\end{trivlist}}

\begin{document}


\title{\vspace{-2cm} Econ 136: Problem Set 1}
\author{Franklin She}

\maketitle
 
\begin{problem}{1}
    \hfill
    \begin{enumerate}
        \item \hfill
                \begin{align*}
                    Cov(X, Y) &= \E[(X - \mu_X)(Y - \mu_Y)] \\
                    &= \E[XY - X\mu_Y - Y\mu_X + \mu_X\mu_Y] \\
                    &= \E[XY] - \E[X\mu_Y] - \E[Y\mu_X] + \E[\mu_X\mu_Y] \\
                    &= \E[XY] - \mu_Y\E[X] - \mu_X\E[Y] + \mu_X\mu_Y \\
                    &= \E[XY] - \mu_X\mu_Y
                \end{align*}
        \item \hfill
                \begin{align*}
                    Cov(X, Y) &= \E[XY] - \mu_X\mu_Y \\
                    &= \E[X^3] - \mu_X\mu_Y \\
                    &= \E[X^3] - \E[X]\E[X^2] \\
                    &= 0 - 0 \E[X^2] \\
                    &= 0
                \end{align*}
    \end{enumerate}
\end{problem}

\begin{problem}{2}
    \hfill
    \begin{align*}
        Var(\mathbf{X}) &= \E[(\mathbf{X} - \E[\mathbf{X}])(\mathbf{X} - \E[\mathbf{X}])^T] \\
                        &=\E[(\mathbf{X} - \E[\mathbf{X}])(\mathbf{X}^T - \E[\mathbf{X}]^T)] \\
                        &= \E[\mathbf{X}\mathbf{X}^T - \mathbf{X}\E[\mathbf{X}]^T - \E[\mathbf{X}]\mathbf{X}^T + \E[\mathbf{X}]\E[\mathbf{X}]^T] \\
                        &= \E[\mathbf{X}\mathbf{X}^T] - \E[\mathbf{X}\E[\mathbf{X}]^T] - \E[\E[\mathbf{X}]\mathbf{X}^T] + \E[\E[\mathbf{X}]\E[\mathbf{X}]^T] \\
                        &= \E[\mathbf{X}\mathbf{X}^T] - \E[\mathbf{X}]\E[\mathbf{X}]^T - \E[\mathbf{X}]\E[\mathbf{X}]^T + \E[\mathbf{X}]\E[\mathbf{X}]^T \\
                        &= \E[\mathbf{X}\mathbf{X}^T] - \E[\mathbf{X}]\E[\mathbf{X}]^T.
    \end{align*}



\end{problem}

\begin{problem}{3}
    \hfill
    \begin{enumerate}
        \item First, we express \((a + b \cdot Y + c \cdot Z) - \tilde{m}(X)\) as:
        \begin{align*}
        (a + b \cdot Y + c \cdot Z) - (a + b \cdot \E[Y | X] + c \cdot \E[Z | X]) &= b \cdot (Y - \E[Y | X]) + c \cdot (Z - \E[Z | X]).
        \end{align*}
        
        Taking the expectation and using the linearity of expectation, we get:
        \begin{align*}
        \E[((a + b \cdot Y + c \cdot Z) - \tilde{m}(X)) \cdot g(X)] &= \E[b \cdot (Y - \E[Y | X]) \cdot g(X) + c \cdot (Z - \E[Z | X]) \cdot g(X)] \\
        &= b \cdot \E[(Y - \E[Y | X]) \cdot g(X)] + c \cdot \E[(Z - \E[Z | X]) \cdot g(X)] \\
        &= b \cdot 0 + c \cdot 0 \\
        &= 0.
        \end{align*}

        \item
            We have shown in part (a) that
            \[ \E[((a + b \cdot Y + c \cdot Z) - \tilde{m}(X)) \cdot g(X)] = 0 \quad \text{ for all } g. \]
            Therefore, by the orthogonality principle proven in class, we can define that $\tilde{m}(X)$ is
            $\E[a + b \cdot Y + c \cdot Z | X]$, the expected value of $a + b \cdot Y + c \cdot Z$ given $X$.
    \end{enumerate}
    
\end{problem}

\begin{problem}{4}`
    \hfill
    \begin{enumerate}
        \item
            To express \(\E[Y | X]\) as a linear function of \(X\), consider the two possible values of \(X\), which are 0 and 1. The conditional expectation of \(Y\) given \(X\) can thus be written as:
            \begin{align*}
            \E[Y | X = 0] &= \beta_0, \\
            \E[Y | X = 1] &= \beta_0 + \beta_1.
            \end{align*}
            Hence, for \(X = 0\) and \(X = 1\), the conditional expectation \(\E[Y | X]\) can be represented as \(\beta_0 + \beta_1X\), where:
            \begin{align*}
            \beta_0 &= \E[Y | X = 0], \\
            \beta_1 &= \E[Y | X = 1] - \E[Y | X = 0].
            \end{align*}

        \item
            We use the law of iterated expectations and the result of part (a),
            \begin{align*}
            \E[Y] &= \E[\E[Y | X]] \\
                  &= \E[\beta_0 + \beta_1X] \\
                  &= \E[\E[Y | X = 0] + (\E[Y | X = 1] - \E[Y | X = 0])X] \\
                  &= \E[\E[Y | X = 0]] + \E[(\E[Y | X = 1] - \E[Y | X = 0])X] \\
                  &= \E[Y | X = 0] + (\E[Y | X = 1] - \E[Y | X = 0])\E[X] \\
                  &= \E[Y | X = 0] + (\E[Y | X = 1] - \E[Y | X = 0])p \\
                  &= p \cdot \E[Y | X = 1] + (1 - p) \cdot \E[Y | X = 0] \\
            \end{align*}
        \item
            We use the law of iterated expectations at the result of part (a),
            \begin{align*}
                \E[XY] &= \E[\E[XY | X]] \\
                       &= \sum_{x} \E[XY | X = x] \cdot \Pr[X = x] \\
                       &= \E[XY | X = 0] \cdot (1-p) + \E[XY | X = 1] \cdot p \\
                       &= p \cdot \E[Y | X = 1] \qquad \text{(by choosing $X=0$ or $X=1$)} \\
            \end{align*}
        \item
            By definition,
            \begin{align*}
            \text{Cov}(X,Y) &= \E[XY] - \E[X]\E[Y], \\
            \text{Var}(X) &= \E[X^2] - (\E[X])^2 = p - p^2.
            \end{align*}

            From (c), solving for $\E[Y | X = 1]$ gives:

            \begin{align*}
                \E[Y | X = 1] &= \frac{\E[XY]}{p}
            \end{align*}

            From (b), solving for $\E[Y | X = 0]$ gives:

            \begin{align*}
                \E[Y]-p\E[Y | X = 1] &= (1-p)\E[Y | X = 0] \\
                \E[Y | X = 0] &= \frac{\E[Y] - p\E[Y | X = 1]}{1-p} \\
                              &= \frac{\E[Y] - \E[XY]/p}{1-p}
            \end{align*}

            Solving for $\beta_1$ gives:

            \begin{align*}
            \beta_1 &= \E[Y | X = 1] - \E[Y | X = 0] \\
                    &= \frac{\E[XY]}{p} - \frac{\E[Y] - \E[XY]/p}{1-p} \\
                    &= \frac{(1-p)\E[XY] - p\E[Y] + \E[XY]}{p(1-p)} \\
                    &= \frac{\E[XY] - \E[Y]\cdot p}{p(1-p)} \\
                    &= \frac{\E[XY] - \E[X]\E[Y]}{p(1-p)} \\
                    &= \frac{\text{Cov}(X,Y)}{\text{Var}(X)}
            \end{align*}

    \end{enumerate}
\end{problem}

\begin{problem}{5}
    \hfill
    \begin{enumerate}
        \item \hfill
                \begin{align*}
                    \E[X_1] &= \E[r_{B,1} - r_f] \\
                    &= \E[r_{B,1}] - \E[r_f] \\
                    &= 0.12 - 0.02 \\
                    &= 0.10
                \end{align*}
                \begin{align*}
                    Var(X_1) &= Var(r_{B,1} - r_f) \\
                    &= Var(r_{B,1}) + Var(r_f) - 2Cov(r_{B,1}, r_f) \\
                    &= Var(r_{B,1}) \\
                    &= 0.01 = 0.1^2
                \end{align*}
                By the properties of the normal distribution, we have that $\frac{X_1 - 0.10}{0.1} \sim N(0, 1)$.
        \item \hfill
            \begin{enumerate}
                \item Since we have shown that $\frac{X_1 - 0.10}{0.1} \sim N(0, 1)$, we can express:
                    \begin{align*}
                        Pr[X_1 \leq 0] &= Pr[\frac{X_1 - 0.10}{0.1} \leq \frac{0 - 0.10}{0.1}] \\
                        &= Pr[Z \leq -1] \\
                        &= \Phi(-1)
                    \end{align*}
                \item R code for calculating the probability:

                    \begin{lstlisting}[style=Rstyle]
probability <- pnorm(-1)
print(probability)
                \end{lstlisting}
                    The probability is approximately 0.1587. There is about a 15.87\% chance of observing zero excess return or less in one year. The relatively low probability suggests that having a zero or negative excess return in one year is not extremely unlikely.
                \item 
                    If we use a significance level of $0.05$, the observed zero excess return in one year is not implausible because 
                    \[0.1587 > 0.05\]
            \end{enumerate}
        \item \hfill
                \begin{enumerate}
                    \item The standard error of the mean for $\bar{X}_4$ is given by $\frac{\sigma}{\sqrt{n}}$, where $\sigma = 0.1$ is the standard deviation of $X_t$ and $n = 4$ is the number of years. Thus, the standard error is $\frac{0.1}{\sqrt{4}} = \frac{0.1}{2} = 0.05$. Since $\E[\bar{X}_4] = 0.10$, we can normalize $\bar{X}_4$ as follows:
                        \begin{align*}
                            Pr[\bar{X}_4 \leq 0] &= Pr\left[\frac{\bar{X}_4 - 0.10}{0.05} \leq \frac{0 - 0.10}{0.05}\right] \\
                            &= Pr[Z \leq -2] \\
                            &= \Phi(-2)
                        \end{align*}
                    \item Using R, the calculation is performed as follows:
                        \begin{lstlisting}[style=Rstyle]
        probability <- pnorm(-2)
        print(probability)
                        \end{lstlisting}
                        The probability is approximately 0.0228. There is about a 2.28\% chance of observing zero or negative average excess return across four years. This low probability suggests that observing such an outcome is quite unlikely.
                    \item 
                    If we use a significance level of $0.05$, the observed zero excess return in one year is implausible because 
                    \[0.0228 < 0.05\]
                \end{enumerate}

            \item \hfill
                \begin{enumerate}
                    \item 
                        The probability in question can be decomposed into two tails of the distribution:
                    \begin{align*}
                        Pr[|\bar{X}_4 - 0.10| \geq 0.10] &= Pr[\bar{X}_4 \leq 0] + Pr[\bar{X}_4 \geq 0.20] \\
                        &= \Phi\left(\frac{0 - 0.10}{0.05}\right) + \left(1 - \Phi\left(\frac{0.20 - 0.10}{0.05}\right)\right) \\
                        &= \Phi(-2) + (1 - \Phi(2)).
                    \end{align*}
                \item Using R, we calculate this probability as follows:
                    \begin{lstlisting}[style=Rstyle]
    probability <- pnorm(-2) + (1 - pnorm(2))
    print(probability)
                    \end{lstlisting}
                    This will yield a probability of approximately 0.0455. There is about a 4.55\% chance of observing an average excess return either 0.10 below or above the asserted expected excess return of 0.10 in one year. 
            \end{enumerate}
    \end{enumerate}
\end{problem}
\end{document}
