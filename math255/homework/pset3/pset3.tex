\documentclass[10pt]{article}
 
\usepackage[margin=1in]{geometry} 
\usepackage{amsmath,amsthm,amssymb, graphicx, multicol, array, bbm}
 
\newcommand{\N}{\mathbb{N}}
\newcommand{\Z}{\mathbb{Z}}
\newcommand{\R}{\mathbb{R}}
\newcommand{\C}{\mathbb{C}}
\newcommand{\A}{\mathcal{A}}
\newcommand{\F}{\mathbb{F}}
\newcommand{\Q}{\mathbb{Q}}
 
\newenvironment{problem}[2][Problem]{\begin{trivlist}
\item[\hskip \labelsep {\bfseries #1}\hskip \labelsep {\bfseries #2.}]}{\end{trivlist}}

\begin{document}


\title{Problem Set 3}
\author{Math 255: Analysis I}
\date{Due: Thursday, Feb 8th at 11:59pm EST}

\maketitle

\begin{problem}{1}
	Prove that the complex field $ \C=\{a+bi : a,b \in \R \} $, with $ i^2=-1 $, does not support an order relation which makes it an ordered field.
    \begin{proof}
        Assume in contradiction that $ \C $ supports an order relation which makes it an ordered field. Consider $i \in \C$. By definition, $i \neq 0$. Then, by trichotomy, either $i > 0$ or $i < 0$.
        \begin{itemize}
            \item If $i > 0$, then $i \cdot i = i^2 = -1 > 0$ by order respects multiplication. This is a contradiction, as $-1 < 0$.
            \item If $i < 0$, then we have $-i > 0$. Then, $-i \cdot -i = i^2 = -1 > 0$ by order respects multiplication. This is a contradiction, as $-1 < 0$.
        \end{itemize}
        Both cases lead to a contradiction, so $ \C $ does not support an order relation which makes it an ordered field.
    \end{proof}
\end{problem}
\medskip

\newpage

\begin{problem}{2} 
	In this problem we will discuss a construction of $ \R $ using Dedekind cuts. 
	
	\noindent A \emph{cut set} in $ \Q $ is a subset $ \alpha \subset \Q $ satisfying:
	\begin{itemize}
		\item $ \alpha \neq \emptyset $ and $ \alpha \neq \Q $;
		\item if $ q \in \alpha $ and $ p < q $ then $ p \in \alpha $; and
		\item $ \alpha $ has no maximum, that is, for all $ q \in \alpha $ there exists $ q' \in \alpha $ with $ q < q' $.
	\end{itemize}
	
	\noindent Consider 
	\[ R = \{ \alpha \subset \Q : \alpha \text{ is a cut set} \}, \]
	the set of all cut sets of $ \Q $. Moreover, we define the relation $ < $ on $ R $ by  
	\[ \alpha < \beta \quad \text{if and only if } \quad \alpha \subsetneq \beta \qquad \text{for all } \alpha,\beta \in R. \] 
	\begin{enumerate}
		\item Prove that $ R $ with this relation is an ordered set (i.e.~show this relation satisfies the trichotomy and transitivity properties).
            \begin{proof}
                \textbf{Trichotomy:} Let $\alpha, \beta \in R$.
                \begin{enumerate}
                    \item If $\alpha < \beta$, then $\alpha \subsetneq \beta$.
                        This implies that $\exists x \in \beta$ such that $x \notin \alpha$. Then, $\beta \nsubseteq \alpha$, so $\alpha \not> \beta$ and $\alpha \neq \beta$.
                    \item If $\alpha = \beta$, then $\alpha \not\subsetneq \beta$ and $\beta \not\subsetneq \alpha$. Therefore, $\alpha \not < \beta$ and $\alpha \not > \beta$.
                    \item If $\alpha > \beta$, then $\beta < \alpha$. Then, $\beta \subsetneq \alpha$. This implies that $\exists x \in \alpha$ such that $x \notin \beta$. Then, $\alpha \nsubseteq \beta$, so $\alpha \not> \beta$ and $\alpha \neq \beta$.
                \end{enumerate}

                \textbf{Transitivity:} Let $\alpha, \beta, \gamma \in R$ such that $\alpha < \beta$ and $\beta < \gamma$. Then, $\alpha \subsetneq \beta$ and $\beta \subsetneq \gamma$. This implies that $\alpha \subsetneq \gamma$, so $\alpha < \gamma$.
            \end{proof}
		\item Prove that $ R $ with this order has the least upper bound property. That is, prove that any non-empty $ A \subset R $ having an upper bound also has a supremum.
             \begin{proof}
                 Let $A \subset R$ be non-empty and have an upper bound. Let \[S = \bigcup_{\alpha \in A} \alpha.\]
                 In other words, $S$ is the set of all rationals that belong to at least one cut set in $A$. We will show that $S$ is a cut set and that it is the supremum of $A$.

                 \textbf{$S$ is a cut set:}
                 \begin{itemize}
                     \item $S \neq \emptyset$: Since $A$ is non-empty, there exists $\alpha \in A$. $\alpha$ is not empty by definition of cut set. Then, $\exists q \in \alpha$, so $q \in S$.
                        \item $S \neq \Q$: Since $A$ has an upper bound, there exists $\beta \in R$ such that $\alpha \leq \beta$ for all $\alpha \in A$. Because $\beta$ is a cut set, $\beta \neq \Q$, so $\exists q \in \Q$ where $q \not\in \beta$. We also know that $q$ is greater than any element in $\beta$, and by transitivity, $q$ is greater than any element of any set $A$.
                            Then, $q \not\in S$, so $S \neq \Q$.
                        \item If $q \in S$ and $p < q$, then $p \in S$: Let $q \in S$ and $p < q$. Then, $q \in \alpha$ for some $\alpha \in A$. Since $\alpha$ is a cut set, $p \in \alpha$, so $p \in S$.
                        \item $S$ has no maximum: For any $q \in S$, there exists $\alpha \in A$ such that $q \in \alpha$. Since $\alpha$ is a cut set, there exists $q' \in \alpha$ such that $q < q'$. Then, $q' \in S$.
                 \end{itemize}

                 \textbf{$S$ is the least upper bound of $A$:}

                 By our construction of $S$, we have that for all $\alpha \in A$, $\alpha \subset S$, so $S$ is an upper bound of $A$. Now, we will show that $S$ is the least upper bound of $A$.
                 Suppose in contradiction that there exists $T$ such that $T < S$ and $T$ is an upper bound of $A$. Then, $T \subsetneq S$.
                 This implies that there exists $q \in S$ such that $q \notin T$. Since $q \in S$, there exists $\alpha \in A$ such that $q \in \alpha$. Since $T$ is an upper bound of $A$, $\alpha \subset T$. Then, $q \in T$, a contradiction. Thus, $S$ is the least upper bound of $A$.
             \end{proof}
		\item Let $ \alpha,\beta \in R $ be two cuts. Prove that
		\[ \alpha+\beta = \{p+q : p \in \alpha , q \in \beta \} \]
		is also a cut set of $ \Q $. Hence we can define an addition operation on $ R $.
        \begin{proof}
            Let us call $\alpha + \beta = \gamma$. We will show that $\gamma$ is a cut set of $\Q$.
            \begin{itemize}
                \item $\gamma \neq \emptyset$: Since $\alpha$ and $\beta$ are cut sets, there exists $p \in \alpha$ and $q \in \beta$. Then, $p+q \in \gamma$.

                \item $\gamma \neq \Q$: Since $\alpha$ and $\beta$ are cut sets, $\alpha \neq \Q$ and $\beta \neq \Q$. Then, there exists $p', q' \in \Q$ such that $p' \not\in \alpha$ and $q' \not\in \beta$. Now consider some $p \in \alpha$, and $q \in \beta$. By definition of a cut set, $p < p'$ and $q < q'$, which implies $p + q < p' + q'$. Thus, $p' + q' \not\in \gamma$.

                \item If $r \in \gamma$ and $s < r$, then $s \in \gamma$: For $r = p + q$ with $p \in \alpha$ and $q \in \beta$, if $s < r$, then $s < p + q$. Since $\alpha$ and $\beta$ are cut sets, if $p' < p$ for some $p' \in \mathbb{Q}$, then $p' \in \alpha$; similarly for $q' < q$. Therefore, we can always find $p' \in \alpha$ and $q' \in \beta$ such that $s = p' + q'$, thus $s \in \gamma$.
    
                \item $\gamma$ has no maximum: Suppose $r \in \gamma$, where $r = p + q$ for some $p \in \alpha$ and $q \in \beta$. Because $\alpha$ and $\beta$ have no maximum, there exists $p' > p$ in $\alpha$ and $q' > q$ in $\beta$. Thus, $r' = p' + q' > r$ is also in $\gamma$, proving $\gamma$ has no maximum.
                
            \end{itemize}
        \end{proof}
		\item How could you identify $ \Q $ as a subset of $ R $?
            \begin{proof}
                For any rational number $q \in \Q$, we define the cut set
                \[ \alpha_q = \{ p \in \Q : p < q \}. \]
                This is a cut set because
                \begin{itemize}
                    \item $\alpha_q \neq \emptyset$: $q-1 \in \alpha_q$. ($\Q$ has no minimum).
                    \item $\alpha_q \neq \Q$: There exists $p \in \Q$ such that $p > q$, so $p \notin \alpha_q$.
                    \item If $p \in \alpha_q$ and $r < p$, then $r \in \alpha_q$: If $r < p$, then $r < q$, so $r \in \alpha_q$.
                    \item $\alpha_q$ has no maximum: We apply denseness of $\Q$ in $\R$ (proven in class) to use the denseness of $\Q$ in $\Q$ (proof is identical) to state that for any $p \in \alpha_q$, we can find an $r \in \Q$, such that $p < r < q$. Thus, $\alpha_q$ has no maximum.
                \end{itemize}
                Then, we can identify $\Q$ as a subset of $R$ by the map $q \mapsto \alpha_q$.
            \end{proof}
		\item \emph{(Not for submission)} How would you define multiplication in $ R $? Try to convince yourselves that these operations define an ordered field structure on $ R $, yielding a construction of $ \R $.
	\end{enumerate}
\end{problem}
\medskip

\newpage

\begin{problem}{3}
	Let $ A,B \subset \R $ be two subsets which are bounded above. Prove:
	\begin{enumerate}
		\item $ \sup(A \cup B) = \max(\sup A, \sup B) $.
            \begin{proof}
                We will show that $\max(\sup A, \sup B)$ is an upper bound for $A \cup B$ and that it is the least upper bound. This is sufficient to show that $\sup(A \cup B) = \max(\sup A, \sup B)$, because we have shown in class that the least upper bound is unique.

                Let $a \in A \cup B$. Then, $a \in A$ or $a \in B$. If $a \in A$, then $a \leq \sup A \leq \max(\sup A, \sup B)$. If $a \in B$, then $a \leq \sup B \leq \max(\sup A, \sup B)$. Thus, $\max(\sup A, \sup B)$ is an upper bound for $A \cup B$.

                Now, let $u$ be an upper bound for $A \cup B$. Then, for all $a \in A \cup B$, $a \leq u$. If $a \in A$, then $a \leq \sup A \leq u$. If $a \in B$, then $a \leq \sup B \leq u$.
                Thus, $\sup A \leq u$ and $\sup B \leq u$. Then, $\max(\sup A, \sup B) \leq u$. Thus, $\max(\sup A, \sup B)$ is the least upper bound for $A \cup B$.
                
            \end{proof}
		\item If $ A \cap B \neq \emptyset $ then $ \sup(A \cap B) \leq \min(\sup A, \sup B) $. Give an example where equality does not hold.
            \begin{proof}
                Let $s_A = \sup A$ and $s_B = \sup B$, given by the LUB property.
                Since $s_A$ and $s_B$ are the least upper bounds of $A$ and $B$, respectively, it follows that for all $a \in A$, $a \leq s_A$, and for all $b \in B$, $b \leq s_B$.
                
                \textbf{Elements of $A \cap B$:} For any element $x \in A \cap B$, $x$ is both an element of $A$ and an element of $B$. Therefore, $x \leq s_A$ and $x \leq s_B$. This means that $x$ is less than or equal to both $s_A$ and $s_B$, and hence $x \leq \min(s_A, s_B)$.
                
                \textbf{Supremum of $A \cap B$:} Since every element $x \in A \cap B$ satisfies $x \leq \min(s_A, s_B)$, it follows that $\min(s_A, s_B)$ is an upper bound for $A \cap B$. To prove that $\sup(A \cap B) \leq \min(s_A, s_B)$, we observe that the supremum of $A \cap B$ is the least upper bound of $A \cap B$. Since $\min(s_A, s_B)$ is an upper bound for $A \cap B$, and by definition of supremum, no number greater than $\sup(A \cap B)$ can serve as an upper bound, it follows that $\sup(A \cap B)$ must be less than or equal to this common upper bound, $\min(s_A, s_B)$.

                \textbf{Example where equality does not hold:}
                Let $A = \{ \frac{n-1}{n} : n \in \N \}$ and $B = \{ \frac{n-1}{2n} : n \in \N \}$. Then, $A \cap B = \{0\}$. We have $\sup A = 1$ and $\sup B = 1/2$. Then, $\sup(A \cap B) = 0 < 1/2 = \min(\sup A, \sup B)$.
            \end{proof}
		\item The set $ -A=\{ -x: x \in A \} $ is bounded below and satisfies $ \inf(-A)=-\sup A $.
            \begin{proof}
                First, we show that $-A$ is bounded below. By assumption, $A$ is bounded above, meaning $\exists u \in \R$ such that $\forall a \in A$, $a \leq u$.
                By definition of inequality, there exists a $x$ such that $a + x = u$ for all $a \in A$.
                Then, we can use the additive inverses and the cancellation law to arrive at $-a = -u + x$ for all $a \in A$. Therefore, $\forall -a \in -A$, $-a \geq -u$. Thus, $-A$ is bounded below by $-u$.
                
                Next, we show that $\inf(-A) = -\sup A$ by showing that $-\sup A$ is a lower bound for $-A$ and there exists lower bound of $-A$ in $\R$ greater than $-\sup A$.
                Let $s = \sup A$, which we obtain by the LUB property of $\R$. By $s$ being an upper bound, we reason similarly to the previous paragraph that $-s$ is a lower bound for $-A$.

                Assume in contradiction there exists $y> -s$ that is a lower bound for $-A$. This implies that for all $x \in -A$, we have $-x\geq y$. Thus, $x \leq -y$, which means $-y$ is an upper bound for $A$. However, since $y>-s$, we have $-y<s$, a contradiction of our assumption that $s = \sup A$. Therefore, no such $y > -s$ can exist, meaning that $-s$ is the greatest lower bound of $A$, i.e. $inf(-A) = -\sup A$. 
            \end{proof}
            \item The set $ A+B=\{a+b: a \in A, b \in B \} $ is bounded above and satisfies $ \sup(A+B)=\sup A + \sup B $.
                \begin{proof}
                    Given that $A$ and $B$ are both bounded above, there exist real numbers $M$ and $N$ such that for all $a \in A$, $a \leq M$, and for all $b \in B$, $b \leq N$. Consider any element $x \in A+B$, where $x = a + b$ for some $a \in A$ and $b \in B$. It follows that $x \leq M + N$, demonstrating that $A+B$ is bounded above by $M + N$.

                    $\sup(A+B)$ is an upper bound for $A+B$: Let $s_A = \sup A$ and $s_B = \sup B$. For any $a \in A$ and $b \in B$, we have $a \leq s_A$ and $b \leq s_B$. Therefore, for any element $x = a + b \in A+B$, we have $x = a + b \leq s_A + s_B$. This shows that $s_A + s_B$ is an upper bound for $A+B$.

                    $\sup(A+B)$ is the least upper bound: Suppose there exists $L < \sup A + \sup B$ such that for all $a \in A$ and $b \in B$, $a + b \leq L$. Given $s_A = \sup A$ and $s_B = \sup B$, by definition of supremum, for any $\epsilon > 0$, there exist $a' \in A$ and $b' \in B$ such that:

\begin{itemize}
    \item $a' > s_A - \frac{\epsilon}{2}$
    \item $b' > s_B - \frac{\epsilon}{2}$
\end{itemize}

Choose $\epsilon = \sup A + \sup B - L > 0$. Such an $\epsilon$ exists because we assumed $L < \sup A + \sup B$. Then, for the chosen $a'$ and $b'$, we have:

$$a' + b' > s_A - \frac{\epsilon}{2} + s_B - \frac{\epsilon}{2} = s_A + s_B - \epsilon = L$$
This is a contradiction because we assumed $L$ is an upper bound for $A+B$, which would mean $a' + b' \leq L$ for all $a' \in A$ and $b' \in B$. However, we have found $a', b'$ such that $a' + b' > L$. Therefore, our initial assumption that there exists an upper bound $L < \sup A + \sup B$ for $A+B$ is false.
                \end{proof}
	\end{enumerate}
\end{problem}

\newpage

\begin{problem}{4}
	For each of the following subsets in $ \R $
	\[ A=\left\{ \frac{(-1)^n \cdot n}{n+1} : n \in \N \right\} \quad,\quad B=\left\{ \frac{(-1)^n}{n+1} : n \in \N \right\} \quad \text{, and}\quad C=\left\{ \frac{1}{(q-1)^2} : q \in \Q, q \neq 1 \right\}  \]
	answer the following (justify your claims):
	\begin{enumerate}
		\item Is the set bounded above\footnote{i.e.~has an upper bound.}? bounded below?
		\item If bounded above what is its supremum? If bounded below what is its infimum?
		\item Does the set have a maximum? minimum?
	\end{enumerate}

    \subsection*{Set $A$}
\begin{enumerate}
    \item \textbf{Boundedness:} The set $A = \left\{\frac{(-1)^n \cdot n}{n+1} : n \in \mathbb{N}\right\}$ is bounded above by $1$ and below by $-1$. For any $n \in \mathbb{N}$, we indeed have $-1 < \frac{(-1)^n \cdot n}{n+1} < 1$. This is because the magnitude of the numerator is always less than the magnitude of the denominator, ensuring the fraction's absolute value is strictly less than $1$.
    
    \item \textbf{Supremum and Infimum:} To prove $\sup A = 1$ and $\inf A = -1$, assume for contradiction there exists an upper bound $U < 1$ or a lower bound $L > -1$. For any $\epsilon > 0$, choose $n \in \mathbb{N}$ such that $\frac{1}{n+1} < \epsilon$. Then, $\frac{n}{n+1} > 1 - \epsilon$, contradicting $U$ as an upper bound. Similarly, we can show $L > -1$ leads to a contradiction, thus $\sup A = 1$ and $\inf A = -1$.
    
    \item \textbf{Maximum and Minimum:} Since for every $n \in \mathbb{N}$, there exists $n+1$ such that $\frac{(-1)^{n+1} \cdot (n+1)}{n+2}$ is closer to either $1$ or $-1$, $A$ does not contain its supremum or infimum as elements. Hence, $A$ has no maximum or minimum.
\end{enumerate}

\subsection*{Set $B$}
\begin{enumerate}
    \item \textbf{Boundedness:} The set $B = \left\{\frac{(-1)^n}{n+1} : n \in \mathbb{N}\right\}$ is bounded above by $1$ and below by $-1$ for similar reasons as $A$, with the absolute value of all elements bounded by $1$ due to the denominator being larger than the numerator in absolute value.
    
    \item \textbf{Supremum and Infimum:} $\sup B = \frac{1}{3}$ and $\inf B = -\frac{1}{2}$ are demonstrated by directly evaluating the sequence for $n=1$ and $n=0$, respectively. No element for $n > 1$ can exceed these bounds due to the increasing denominator, establishing these as the supremum and infimum.
    
    \item \textbf{Maximum and Minimum:} As $\frac{1}{3}$ and $-\frac{1}{2}$ are explicitly present in $B$ for $n=1$ and $n=0$ respectively, they are the maximum and minimum of $B$.
\end{enumerate}

\subsection*{Set $C$}
\begin{enumerate}
    \item \textbf{Boundedness:} The set $C = \left\{\frac{1}{(q-1)^2} : q \in \mathbb{Q}, q \neq 1\right\}$ is unbounded above because as $q$ approaches $1$, the denominator approaches $0$, causing the fraction to increase without bound. However, $C$ is bounded below by $0$ since the fraction is always positive.
    
    \item \textbf{Infimum:} We show $\inf C = 0$ using an $\epsilon$-$\delta$ argument. For any $\epsilon > 0$, choose $q$ such that $0 < |q - 1| < \sqrt{\frac{1}{\epsilon}}$. Then, $\frac{1}{(q-1)^2} > \epsilon$, indicating the fraction approaches $0$ as $q$ approaches $1$, but never reaches $0$.
    
    \item \textbf{Maximum and Minimum:} $C$ has no maximum or minimum. The minimum value of $0$ is approached but never reached because the
        numerator is always $1$ and the denominator is always positive, ensuring the fraction is always positive.
        The lack of an upper bound directly implies no maximum.
\end{enumerate}

\end{problem}
\medskip



\end{document}
