\documentclass[10pt]{article}
 
\usepackage[margin=1in]{geometry} 
\usepackage{amsmath,amsthm,amssymb, graphicx, multicol, array, bbm}
 
\newcommand{\N}{\mathbb{N}}
\newcommand{\Z}{\mathbb{Z}}
\newcommand{\R}{\mathbb{R}}
\newcommand{\C}{\mathbb{C}}
\newcommand{\A}{\mathcal{A}}
\newcommand{\F}{\mathbb{F}}
\newcommand{\Q}{\mathbb{Q}}
\newcommand{\abs}[1]{\left| #1 \right|}
 
\newenvironment{problem}[2][Problem]{\begin{trivlist}
\item[\hskip \labelsep {\bfseries #1}\hskip \labelsep {\bfseries #2.}]}{\end{trivlist}}

\begin{document}


\title{Problem Set 4}
\author{Math 255: Analysis I}
\date{Due: Thursday, Feb 15th at 11:59pm EST}

\maketitle

%\begin{problem}{1}
%	Prove that a union of countably many countable sets is countable. That is, if $ S_1,S_2,... $ are countable sets for all $ n \in \N $ then $ \bigcup_{n=1}^\infty S_n $ is a countable set.
%\end{problem}

%\begin{problem}{1}
%	Given a set $ A $, we denote by $ \mathcal{P}(A) $ the set of all subsets of $ A $:
%	\[ \mathcal{P}(A) = \{ E : E \subset A \}. \]
%	In this problem we will prove that $ A $ and $ \mathcal{P}(A) $ do \textbf{not} have equal cardinality, for any $ A $.
%	\begin{enumerate}
%		\item As a warm-up, consider $ A=\{1,2,3\} $. What size is $ \mathcal{P}(A) $?
%		\item Let $ A $ be any set. Assume in contradiction that there exists a bijection 
%		\[ \phi: A \to \mathcal{P}(A), \]
%		and consider the set
%		\[ E = \{ a \in A : a \in \phi(a) \}. \]
%		Prove that there exists no element $ z \in A $ with $ \phi(z)=E $. Deduce that $ |A| \neq |\mathcal{P}(A)| $.
%		\item Conclude that the set of all subsets of $ \N $ is uncountable.
%	\end{enumerate}
%\end{problem}
%\medskip

\begin{problem}{1}
	In this problem we give another proof that $ \mathcal{P}(\N) $, the set of all subsets of $ \N $, is uncountable.
	\begin{enumerate}
		\item Consider the set of all binary sequences:
		\[ \{0,1\}^{\N} = \{ (a_1,a_2,a_3,...) : a_n \in \{0,1\} \text{ for all } n \in \N \}. \]
		Find a bijection between $ \cal{P}(\N) $ and $ \{0,1\}^{\N} $.
		\item Use Cantor's diagonal argument to show that $ \{0,1\}^{\N} $ is uncountable, and conclude that $ \cal{P}(\N) $ is uncountable.
	\end{enumerate}
\end{problem}

\begin{problem}{2} 
A real number $x$ is called \emph{algebraic} if it is a root of an integer polynomial. That is, if there exists an $n \in \N$ and integers $a_0, \dots, a_n$, not all zero, such that
\begin{equation*}
a_0 x^n + a_1 x^{n-1} + \cdots + a_{n-1} x + a_n = 0.
\end{equation*}

\begin{enumerate}
	\item
	Prove that $\sqrt{5}$ and $\sqrt{2 + \sqrt{3}}$ are algebraic.
	
	\item 
	Prove that the set of all algebraic real numbers is countable.  
	
	\noindent (Hint: for every positive integer $N$, there are
	only finitely many ways to choose integers $n, a_0, a_1, \dots, a_n$ with $n + \abs{a_0} + \abs{a_1} + \cdots + \abs{a_n} = N$.  You may use the fact that each polynomial has a finite number of roots.)
	
	\item Prove that there exist real numbers which are not algebraic.
\end{enumerate}
\end{problem}
\medskip

\begin{problem}{3}
	Let $X$ be any set, let and $d:X\times X\to \R$ be the discrete metric, defined by
	\[
	d(x,y)=\begin{cases}0&x=y \, \\ 1&x\neq y\end{cases}
	\]
	for all $x,y\in X$.
	\begin{enumerate}
		\item Prove that, with this distance function, $X$ is a metric space.
		\item For any $x\in X$, what is $N_\epsilon(x)$ when $\epsilon = \frac{1}{2}$, $1$, and $2$?
		\item Which subsets of $X$ are open?
	\end{enumerate}
\end{problem}

\vfill\hfill \emph{turn page}$ \longrightarrow $
\newpage

\begin{problem}{4}
	Let $E$ be a subset of a metric space. Define the \emph{interior} of $E$, denoted $E^\circ$, to be the set of all interior points of $E$.
	\begin{enumerate}
		\item Prove that $E^\circ$ is always open.
		\item Prove that $E$ is open if and only if $E^\circ = E$.
		\item Prove that, if $G$ is an open subset of $E$, then $G \subset E^\circ$.
		\item Prove that $ \Q \subset \R $ has an empty interior.
		\item Prove that $ \R \smallsetminus \Q $, the complement of $ \Q $ in $ \R $, also has an empty interior.
	\end{enumerate}
\end{problem}


\vspace{3cm}\hfill \emph{Good luck!}

\end{document}