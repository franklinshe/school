\documentclass[10pt]{article}
 
\usepackage[margin=1in]{geometry} 
\usepackage{amsmath,amsthm,amssymb, graphicx, multicol, array, bbm}
 
\newcommand{\N}{\mathbb{N}}
\newcommand{\Z}{\mathbb{Z}}
\newcommand{\R}{\mathbb{R}}
\newcommand{\C}{\mathbb{C}}
\newcommand{\A}{\mathcal{A}}
\newcommand{\F}{\mathbb{F}}
\newcommand{\Q}{\mathbb{Q}}
\newcommand{\abs}[1]{\left| #1 \right|}
 
\newenvironment{problem}[2][Problem]{\begin{trivlist}
\item[\hskip \labelsep {\bfseries #1}\hskip \labelsep {\bfseries #2.}]}{\end{trivlist}}

\begin{document}


\title{Problem Set 4}
\author{Math 255: Analysis I}
\date{Due: Thursday, Feb 15th at 11:59pm EST}

\maketitle

\begin{problem}{1}
	In this problem we give another proof that $ \mathcal{P}(\N) $, the set of all subsets of $ \N $, is uncountable.
	\begin{enumerate}
		\item Consider the set of all binary sequences:
		\[ \{0,1\}^{\N} = \{ (a_1,a_2,a_3,...) : a_n \in \{0,1\} \text{ for all } n \in \N \}. \]
		Find a bijection between $ \cal{P}(\N) $ and $ \{0,1\}^{\N} $.
        \begin{proof}
            We can define a function $f: \{0,1\}^{\N} \to \mathcal{P}(\N)$ by
            \[f((a_1,a_2,a_3,...)) = \{n \in \N : a_n = 1\}.\]
            \textbf{Injectivity:} Suppose $f((a_1,a_2,a_3,...)) = f((b_1,b_2,b_3,...))$. Then $\{n \in \N : a_n = 1\} = \{n \in \N : b_n = 1\}$. This means that $a_n = b_n$ for all $n \in \N$, so $(a_1,a_2,a_3,...) = (b_1,b_2,b_3,...)$. Thus, $f$ is injective.

            \textbf{Surjectivity:} Let $E \in \mathcal{P}(\N)$. Then we can define a sequence $(a_1,a_2,a_3,...)$ by
            \[a_n = \begin{cases} 1 & \text{if } n \in E \\ 0 & \text{if } n \notin E \end{cases}.\]
            Then $f((a_1,a_2,a_3,...)) = E$. Thus, $f$ is surjective.
        \end{proof}
		\item Use Cantor's diagonal argument to show that $ \{0,1\}^{\N} $ is uncountable, and conclude that $ \cal{P}(\N) $ is uncountable.
            \begin{proof}
            Assume in contradiction that  $ \{0,1\}^{\N} $ is countable. Hence, $ \{0,1\}^{\N} = \{x_1, x_2, \ldots \} $. We can construct $y \in \{0,1\}^{\N} $ not appearing in the list.
            Choose $y$ such that its $n$-th entry is different from the $n$-th entry of $x_n$. Then $y \neq x_n$ for all $n$, so $y$ is not in the list. This is a contradiction, so $ \{0,1\}^{\N} $ is uncountable. Since $ \{0,1\}^{\N} $ is uncountable and there is a bijection between $ \{0,1\}^{\N} $ and $ \mathcal{P}(\N) $, we conclude that $ \mathcal{P}(\N) $ is uncountable.
            \end{proof}
	\end{enumerate}
\end{problem}

\newpage

\begin{problem}{2} 
A real number $x$ is called \emph{algebraic} if it is a root of an integer polynomial. That is, if there exists an $n \in \N$ and integers $a_0, \dots, a_n$, not all zero, such that
\begin{equation*}
a_0 x^n + a_1 x^{n-1} + \cdots + a_{n-1} x + a_n = 0.
\end{equation*}

\begin{enumerate}
	\item
	Prove that $\sqrt{5}$ and $\sqrt{2 + \sqrt{3}}$ are algebraic.

    \begin{proof}
        \hfill

        \textbf{$\sqrt{5}$ is algebraic:} Let $n = 2$, $a_0 = 1$, $a_1 = 0$, and $a_2 = -5$. 
        \begin{align*}
            a_0 x^2 + a_1 x + a_2 &= 0 \\
            1 x^2 + 0 x - 5 &= 0 \\
            x^2 - 5 &= 0 \\
            \sqrt{5}^2 - 5 &= 0 \\
            5 - 5 &= 0
        \end{align*}

        \textbf{$\sqrt{2 + \sqrt{3}}$ is algebraic:} Let $n = 4$, $a_0 = 1$, $a_1 = 0$, $a_2 = -4$, $a_3 = 0$, and $a_4 = 1$.
        \begin{align*}
            a_0 x^4 + a_1 x^3 + a_2 x^2 + a_3 x + a_4 &= 0 \\
            1 x^4 + 0 x^3 - 4 x^2 + 0 x + 1 &= 0 \\
            x^4 - 4 x^2 + 1 &= 0 \\
            \left(\sqrt{2 + \sqrt{3}}\right)^4 - 4\left(\sqrt{2 + \sqrt{3}}\right)^2 + 1 &= 0 \\
            \left(2 + \sqrt{3}\right)^2 - 4\left(2 + \sqrt{3}\right) + 1 &= 0 \\
            4 + 4\sqrt{3} + 3 - 8 - 4\sqrt{3} + 1 &= 0 \\
            8 - 8 &= 0
        \end{align*}
    \end{proof}
	
	\item 
	Prove that the set of all algebraic real numbers is countable.  
	
	\noindent (Hint: for every positive integer $N$, there are
	only finitely many ways to choose integers $n, a_0, a_1, \dots, a_n$ with $n + \abs{a_0} + \abs{a_1} + \cdots + \abs{a_n} = N$.  You may use the fact that each polynomial has a finite number of roots.)

    \begin{proof}
        Let $A$ be the set of all algebraic real numbers. For each $N \in \N$, let $A_N$ be the set of all algebraic real numbers that are roots of integer polynomials of degree $n$ with coefficients $a_0, a_1, \dots, a_n$ such that $n + \abs{a_0} + \abs{a_1} + \cdots + \abs{a_n} = N$.

        We will first show that each $A_N$ is finite.
        By the hint, for every positive integer $N$, there are only finitely many ways to choose integers $n, a_0, a_1, \dots, a_n$ with $n + \abs{a_0} + \abs{a_1} + \cdots + \abs{a_n} = N$. Thus, $A_N$ is finite.

        Then we show $A = \bigcup_{N \in \N} A_N$. 
        This is true because for any $n, a_0, a_1, \dots, a_n$, there is a unique $N$ such that $n + \abs{a_0} + \abs{a_1} + \cdots + \abs{a_n} = N$.
        So, any algebraic real number is in $A_N$ for some $N \in \N$.

        Finally, by the corollary from class, that any union of countably many finite sets is at most countable, we have that $A$ is countable.
    \end{proof}
	
	\item Prove that there exist real numbers which are not algebraic.
        \begin{proof}
            Assume in contradiction that all real numbers are algebraic.
            That is, for any real number $x$, there exists an $n \in \N$ and integers $a_0, a_1, \dots, a_n$, not all zero, such that
            \[a_0 x^n + a_1 x^{n-1} + \cdots + a_{n-1} x + a_n = 0.\]
            This implies that the set of all real numbers is contained in the set of all algebraic real numbers, which is countable by the previous part.

            By the proposition proved in class (If $S$ is countable and $A \subseteq S$ is infinite, then $A$ is countable), the set of all real numbers is countable.
            This is a contradiction, as we know that the set of all real numbers is uncountable, a corollary shown in class. Thus, there exist real numbers which are not algebraic.
        \end{proof}
\end{enumerate}
\end{problem}

\newpage

\begin{problem}{3}
	Let $X$ be any set, let and $d:X\times X\to \R$ be the discrete metric, defined by
	\[
	d(x,y)=\begin{cases}0&x=y \, \\ 1&x\neq y\end{cases}
	\]
	for all $x,y\in X$.
	\begin{enumerate}
		\item Prove that, with this distance function, $X$ is a metric space.
            \begin{proof}
                We will prove that $(X, d)$ is a metric space by verifying the three properties of a metric space.
                \textbf{Positivity:} Let $x, y \in X$. If $x = y$, then $d(x, y) = 0$. If $x \neq y$, then $d(x, y) = 1$. In either case, $d(x, y) \geq 0$.

                \textbf{Symmetry:} Let $x, y \in X$. If $x = y$ ($\implies y=x$), then $d(x, y) = d(y, x) = 0$. If $x \neq y$ ($\implies y\neq x$), then $d(x, y) = d(y, x) = 1$. In either case, $d(x, y) = d(y, x)$.

                \textbf{Triangle Inequality:} Let $x, y, z \in X$. If $x = y = z$, then $d(x, z) = d(x, y) + d(y, z) = 0$. If $x = y \neq z$, then $d(x, z) = d(x, y) + d(y, z) = 1$. If $x \neq y = z$, then $d(x, z) = d(x, y) + d(y, z) = 1$. If $x \neq y \neq z$, then $d(x, z) = d(x, y) + d(y, z) = 2$. In all cases, $d(x, z) \leq d(x, y) + d(y, z)$.
            \end{proof}
		\item For any $x\in X$, what is $N_\epsilon(x)$ when $\epsilon = \frac{1}{2}$, $1$, and $2$?
            \begin{proof}
\hfill

                \textbf{$\epsilon = \frac{1}{2}$}: Then $N_\epsilon(x) = \{y \in X : d(x, y) < \frac{1}{2}\} = \{x\}$.
                
                \textbf{$\epsilon = 1$}: Then $N_\epsilon(x) = \{y \in X : d(x, y) < 1\} = \{x\}$.

                \textbf{$\epsilon = 2$}: Then $N_\epsilon(x) = \{y \in X : d(x, y) < 2\} = X$.

                $\epsilon = \frac{1}{2}$ and $\epsilon = 1$ give the same set because the only point in $X$ that is within distance $1$ of $x$ is $x$ itself, or $d(x,x) = 1$. For $\epsilon = 2$, every point in $X$ is within distance $2$ of $x$.
            \end{proof}
		\item Which subsets of $X$ are open?
            \begin{proof}
                We will prove that every subset of $X$ is open. Let $E \subset X$. Then for every $x \in E$, we can choose $\epsilon = \frac{1}{2}$, and then $N_\epsilon(x) = \{x\} \subseteq E$. Thus, $E$ is open.
            \end{proof}
	\end{enumerate}
\end{problem}

\newpage

\begin{problem}{4}
	Let $E$ be a subset of a metric space. Define the \emph{interior} of $E$, denoted $E^\circ$, to be the set of all interior points of $E$.
	\begin{enumerate}
		\item Prove that $E^\circ$ is always open.
            \begin{proof}
                A set $E$ is open if all of its points are interior points, so $E^\circ$, as it is the set of all interior points of $E$ is open.
            \end{proof}
		\item Prove that $E$ is open if and only if $E^\circ = E$.
            \begin{proof} Prove both directions.

                ($\implies$) Suppose $E$ is open. Then every point in $E$ is an interior point of $E$, so $E^\circ = E$.

                ($\impliedby$) Suppose $E^\circ = E$. Then every point in $E$ is an interior point of $E$, so $E$ is open.
            \end{proof}

		\item Prove that, if $G$ is an open subset of $E$, then $G \subset E^\circ$.
            \begin{proof}
                Suppose $G$ is an open subset of $E$. Then every point in $G$ is an interior point of $G$. If a point in $G$ is an interior point of $G$, then it is also an interior point of $E$, as $G \subset E$, so $G \subseteq E^\circ$.
            \end{proof}
		\item Prove that $ \Q \subset \R $ has an empty interior.
            \begin{proof}
                To prove that $ \Q $ has an empty interior, we will show that the set of all interior points of $ \Q $ is empty.
                \begin{itemize}
                    \item Let $x \in \Q$.
                Then for every $ \epsilon > 0 $, $N_\epsilon(x) = \{y \in \R : d(x, y) < \epsilon\}$.
                    \item
                We can show that in $N_\epsilon(x)$, there is a point $y \in \R \smallsetminus \Q$.
                    \item
                Let $y = x + \frac{\epsilon}{\sqrt{2}}$. Then $y \in N_\epsilon(x)$, but $y \not\in \Q$.
                    \item
                This means that $N_\epsilon(x) \not\subseteq \Q$. Thus, $x$ is not an interior point of $ \Q $.
                    \item
                Since $x$ was arbitrary, this holds for any $x \in \Q$.
                Thus, $ \Q $ has an empty interior.
                \end{itemize}
            \end{proof}
		\item Prove that $ \R \smallsetminus \Q $, the complement of $ \Q $ in $ \R $, also has an empty interior.
            \begin{proof}
                We can show that $ \R \smallsetminus \Q $ has an empty interior by showing that the set of all interior points of $ \R \smallsetminus \Q $ is empty.
                \begin{itemize}
                    \item
                Let $x \in \R \smallsetminus \Q$.
                Then for every $ \epsilon > 0 $, $N_\epsilon(x) = \{y \in \R : d(x, y) < \epsilon\}$.
                    \item
                We can show that in $N_\epsilon(x)$, there is a point $y \in \Q$. 
                    \item
                We have that $x \in \R$ and $x + \epsilon \in \R$. By the denseness of $ \Q $ in $ \R $ (proven in class), there exists a $y \in \Q$ such that $x < y < x + \epsilon$.
                    \item
                Then $d(x, y) < \epsilon$, so $y \in N_\epsilon(x)$, but $y \not\in \R \smallsetminus \Q$.
                    \item
                This means that $N_\epsilon(x) \not\subseteq \R \smallsetminus \Q$.
                Thus, $x$ is not an interior point of $ \R \smallsetminus \Q $.
                    \item
                Since $x$ was arbitrary, this holds for any $x \in \R \smallsetminus \Q$.
                Thus, $ \R \smallsetminus \Q $ has an empty interior.
                \end{itemize}
            \end{proof}
	\end{enumerate}
\end{problem}


\end{document}
