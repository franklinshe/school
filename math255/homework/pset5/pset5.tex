\documentclass[10pt]{article}
 
\usepackage[margin=1in]{geometry} 
\usepackage{amsmath,amsthm,amssymb, graphicx, multicol, array, bbm}
 
\newcommand{\N}{\mathbb{N}}
\newcommand{\Z}{\mathbb{Z}}
\newcommand{\R}{\mathbb{R}}
\newcommand{\C}{\mathbb{C}}
\newcommand{\A}{\mathcal{A}}
\newcommand{\F}{\mathbb{F}}
\newcommand{\Q}{\mathbb{Q}}
\newcommand{\abs}[1]{\left| #1 \right|}
 
\newenvironment{problem}[2][Problem]{\begin{trivlist}
\item[\hskip \labelsep {\bfseries #1}\hskip \labelsep {\bfseries #2.}]}{\end{trivlist}}

\begin{document}


\title{Problem Set 5}
\author{Math 255: Analysis I}
\date{Due: Thursday, Feb 22nd at 11:59pm EST}

\maketitle

\begin{problem}{1}
	Show that the following subset of $\R^2$ is open\footnote{Whenever not explicitly stated otherwise, $ \R^n $ is considered with the ``natural'' Euclidean metric $ d(v,w)=[\sum_{i=1}^n (v_i-w_i)^2]^{\frac{1}{2}} $.}:
	$$E = \{(x,y) \in \R^2 \, \vert \, x < y\}.$$

    \begin{proof}
        We will prove that for every point in $E$, there exists an $\epsilon$-neighborhood of that point contained within $E$.
        An $\epsilon$-neighborhood of a point $p = (a, b) \in E$ is the set
        $$N_\epsilon(p) = \{q \in R^2 \mid \sqrt{(x-a)^2 + (y-b)^2} < \epsilon\}$$
        We will find an $\epsilon$ such that for all $(x, y) \in N_\epsilon(p)$, $x < y$.
        Let $\epsilon = \frac{b-a}{3} > 0$, because $b > a$.
        Then, every point $(x, y) \in N_\epsilon(p)$ satisfies
        $$\sqrt{(x-a)^2 + (y-b)^2} < \frac{b-a}{3}.$$
        Notice, that this expression implies the following two inequalities:
        \begin{align*}
            |x - a| &< \frac{b-a}{3} \\
            |y - b| &< \frac{b-a}{3}
        \end{align*}
        This in turn implies:
        \begin{align*}
            a - \frac{b-a}{3} < x < a + \frac{b-a}{3} \\
            b - \frac{b-a}{3} < y < b + \frac{b-a}{3}
        \end{align*}
        To prove that $x < y$, we will take the highest possible value of $x$ and the lowest possible value of $y$ and show that $y - x > 0$.
        Let $x = a + \frac{b-a}{3}$ and $y = b - \frac{b-a}{3}$.
        Then, 
        \begin{align*}
            y - x &= b - \frac{b-a}{3} - (a + \frac{b-a}{3}) \\
                  &= b - a - \frac{b-a}{3} - \frac{b-a}{3} \\
                  &= \frac{b-a}{3} > 0. & \text{(since $b > a$)}
        \end{align*}
        Hence, $x < y$ for every point $(x, y) \in N_\epsilon(p)$, implying $N_\epsilon(p) \subseteq E$. This applies to every point $p \in E$. Therefore, every point in $E$ is an interior point, proving $E$ is open in $\R^2$.
    \end{proof}
\end{problem}
\medskip

\newpage

\begin{problem}{2}
	For each of the following subsets of $ \R $ describe all of their interior points and all of their limit points. Determine whether these sets are open, closed or neither. Justify your answers.
	\begin{enumerate}
		\item $ A=\N \cup \left\{ n+\frac{1}{n} : n \in \N \right\} $
            \begin{proof} First, let's enumerate some elements of $A$. Notice that $A$ doesn't contain any irrational numbers.
                $$A = \{1, 2, 2 + \frac{1}{2}, 3, 3 + \frac{1}{3}, 4, 4 + \frac{1}{4}, \dots\}$$
                \textbf{Interior Points:} $A$ has no interior points. We will show that $\forall a \in  A$, $\forall \epsilon > 0$, $\exists x \in N_\epsilon(a)$, such that $x \not\in A$(implying that $N_\epsilon(c) \not\subseteq C$). After choosing any $\epsilon$ for any $a$, we examine the case where $\epsilon$ is rational and the case where $\epsilon$ is irrational. In both cases, we will find an $x \in N_\epsilon(a)$, but $x \not\in A$.
                \begin{enumerate}
                    \item $\epsilon$ is rational: Let $x = a + \frac{\sqrt{2}}{2} \cdot \epsilon$
                    \item $\epsilon$ is irrational: Let $x = a + \frac{1}{2} \cdot \epsilon$
                \end{enumerate}
                We can see that $d(a, x) < \epsilon$, so $x \in N_\epsilon(a)$, but $x$ is irrational, so $x \not\in A$.
                Therefore, $A$ has no interior points.

                \textbf{Limit Points:} $A$ has no limit points, that is, $A$ is composed entirely of isolated points. We will show $\forall a \in A$, $\exists \epsilon > 0$, such that $N_\epsilon(a) \cap A = \{a\}$.

                Let $a \in A$. Either $a = n \in \N$ or $a = n + \frac{1}{n}$ for some $n \in \N$. 
                Notice that the closest point to $a$ in $A$ which is not $a$ is always a distance $\frac{1}{n}$ away. Let's take $\epsilon = \frac{1}{2n}$. Therefore, the $\epsilon$-neighborhood of point $a$ will see no points other points in $A$, showing that $a$ is an isolated point with $\epsilon = \frac{1}{2n}$.

                \textbf{Open/Closed:} $A$ is not open because $A$ has no interior points. $A$ is closed as it contains all its limits points, which is none.
            \end{proof}
		\item $ B=[-1,1] \smallsetminus \left\{ \frac{1}{2n} : n \in \N \right\} $
            \begin{proof}
                \textbf{Interior Points:} The set of interior points of $B$ is $(-1, 1) \smallsetminus \left\{ \frac{1}{2n} : n \in \N \right\} $.
                $B$ is a closed interval with "holes" at $\left\{ \frac{1}{2n} : n \in \N \right\} $. Therefore, the interior points of $B$ are all the points in $(-1, 1)$ that are not in $\left\{ \frac{1}{2n} : n \in \N \right\} $.

                \textbf{Limit Points:} The set of limit points of $B$ is $[-1, 1]$. It is trivial that all points in $B$ are limit points. To see why all points of $\left\{ \frac{1}{2n} : n \in \N \right\} $ are limit points, any point $p$ in this set exists in the interval $(-1, 1)$. Therefore, any $\epsilon$-neighborhood around $p$ will contain (infinitely many) elements of $(-1, 1) \subseteq B$.

                \textbf{Open/Closed:} $B$ is not open because $-1$ and $1$ are not interior points but are in $B$.
                $B$ is not closed because $ \left\{ \frac{1}{2n} : n \in \N \right\} $ are limit points of $B$ not contained in $B$.
                
            \end{proof}
		\item $ C=\left\{ \frac{1}{q^2} : q \in \Q, q \neq 0 \right\} $
            \begin{proof}
                First, let's rewrite the set $C$. Notice that $C$ doesn't contain any irrational numbers.
                $$ C = \left\{\frac{1}{\left(\frac{p}{q}\right)^2} : p, q \in \Z, p, q \neq 0\right\} = \left\{\frac{q^2}{p^2} : p, q \in \Z, p, q \neq 0\right\} $$

                \textbf{Interior Points:} $C$ has no interior points. We will show that $\forall c \in  C$, $\forall \epsilon > 0$, $\exists x \in N_\epsilon(c)$, such that $x \not\in C$ (implying that $N_\epsilon(c) \not\subseteq C$). After choosing any $\epsilon$ for any $c$, we examine the case where $\epsilon$ is rational and the case where $\epsilon$ is irrational. In both cases, we will find an $x \in N_\epsilon(c)$, but $x \not\in C$.
                \begin{enumerate}
                    \item $\epsilon$ is rational: Let $x = c + \frac{\sqrt{2}}{2} \cdot \epsilon$
                    \item $\epsilon$ is irrational: Let $x = c + \frac{1}{2} \cdot \epsilon$
                \end{enumerate}
                We can see that $d(c, x) < \epsilon$, so $x \in N_\epsilon(c)$, but $x$ is irrational, so $x \not\in C$.
                Therefore, $C$ has no interior points.

                \textbf{Limit Points:} The set of limit points of $C$ are the positive real numbers and $0$. This is similar to how the set of limit points of $\Q$ is $\R$. However $C$ is contained to positive squared rationals. Still, we can get arbitrarily close to any positive point in $\R$.
                For any positive real number or $0$, you can find fractions $\frac{q^2}{p^2}$ approaches that real number as closely as desired.

                \textbf{Open/Closed:} $C$ is not open because $C$ has no interior points. $C$ is not closed because it does not contain all its limit points, e.g. $0$.
            \end{proof}
	\end{enumerate}
\end{problem}
\medskip

\newpage
\begin{problem}{3} 
	Let $A_1$, $A_2$, \dots\ be subsets of a metric space $X$.
	
	\begin{enumerate}
		\item If $B_n = \bigcup_{i=1}^n A_i$, prove that $\overline{B}_n = \bigcup_{i=1}^n \overline{A}_i$, for any $n \in {\mathbb N}$.
            \begin{proof}
                Remember that the closure of a set A is defined as the set of all points in $X$ that are either in $A$ or are limit points of $A$. 

                First, we show $\bigcup_{i=1}^n \overline{A}_i \subseteq \overline{B}_n$.

                Let $x \in \bigcup_{i=1}^n \overline{A}_i$.
                This means that there exists some $j \in {1, 2, \dots, n}$ such that $x \in \overline{A}_j$.
                Since $x \in \overline{A}_j$, $x$ is either in $A_j$ or is a limit point of $A_j$.
                \begin{enumerate}
                    \item 
                        If $x$ is in $A_j$, then clearly $x$ is in $B_n = \bigcup_{i=1}^n A_i$, and thus in $\overline{B}_n$.
                    \item
                        If $x$ is a limit point of $A_j$, then every neighborhood of $x$ contains at least one point of $A_j$ different from $x$. Since $A_j \subseteq B_n$, every neighborhood of $x$ also contains at least one point of $B_n$ different from $x$, making $x$ a limit point of $B_n$. Thus, $x \in \overline{B}_n$.
                \end{enumerate}

                This shows that every element of $\bigcup_{i=1}^n \overline{A}_i$ is also an element of $\overline{B}_n$.

                Second, we show $\overline{B}_n \subseteq \bigcup_{i=1}^n \overline{A}_i$

                Let $x \in \overline{B}_n$. This means $x$ is either in $B_n$ or is a limit point of $B_n$.

                \begin{enumerate}
                    \item
                        If $x$ is in $B_n$, then $x$ is in at least one $A_i$ for some $i \in {1, 2, \dots, n}$, and hence $x$ is in at least one $\overline{A}_i$ since $A_i \subseteq \overline{A}_i$.
                    \item
                        If $x$ is a limit point of $B_n$, then every neighborhood of $x$ contains at least one point of $B_n$ different from $x$. Since $B_n = \bigcup_{i=1}^n A_i$, this point must be in at least one $A_i$. Therefore, $x$ must also be a limit point at least this particular $A_i$, and thus $x \in \overline{A}_i$ for some $i$.
                \end{enumerate}

                This shows that every element of $\overline{B}_n$ is also an element of $\bigcup_{i=1}^n \overline{A}_i$.
            \end{proof}
		\item If $B = \bigcup_{i=1}^\infty A_i$, prove that $\overline{B} \supset \bigcup_{i=1}^\infty \overline{A}_i$.
		Show, by an example, that this inclusion can be proper, i.e. it may happen that $\overline{B} \neq \bigcup_{i=1}^\infty \overline{A}_i$.
            \begin{proof}
                Let $x \in \bigcup_{i=1}^\infty \overline{A}_i$.
                This means that $x \in \overline{A}_j$ for some $j$. This implies that $x$ is either in $A_j$ or is a limit point of $A_j$.

                \begin{enumerate}
                    \item 
                        If $x$ is in $A_j$, then clearly $x$ is in $B = \bigcup_{i=1}^\infty A_i$, and thus in $\overline{B}$.
                    \item
                        If $x$ is a limit point of $A_j$, then every neighborhood of $x$ contains at least one point of $A_j$ different from $x$. Since $A_j \subseteq B$, every neighborhood of $x$ also contains at least one point of $B$ different from $x$, making $x$ a limit point of $B$. Thus, $x \in \overline{B}$.
                \end{enumerate}

                This shows that every element of $\bigcup_{i=1}^\infty \overline{A}_i$ is also an element of $\overline{B}$.

                \textbf{Proper Inclusion Example:}

                Consider the metric space $X = \mathbb{R}$ with the usual metric, and let $A_n = \left\{\frac{1}{n}\right\}$ for each $n \in \mathbb{N}$.
                Thus, we have $A_1 = {1}, A_2 = \left\{\frac{1}{2}\right\},$ and so on.
                Then, $$B = \bigcup_{i=1}^\infty A_i = \left\{1, \frac{1}{2}, \frac{1}{3}, \ldots\right\}$$.
                The closure of each $A_n$ is $\overline{A}_n = A_n$ since each $A_n$ contains a single point and is therefore closed.
                So, $$\bigcup_{i=1}^\infty \overline{A}_i = \left\{1, \frac{1}{2}, \frac{1}{3}, \dots\right\}$$
                However, $\overline{B} = B \cup {0}$, because $0$ is a limit point of $B$ and it is not contained in any $A_i$.
                Thus, $\overline{B}$ is strictly larger than $\bigcup_{i=1}^\infty \overline{A}_i$, demonstrating that the inclusion can be proper.

                
            \end{proof}
	\end{enumerate}
\end{problem}
\medskip

\newpage
\begin{problem}{4}
	Given two sets $ A $ and $ B $ in $ \R $, consider the set
	\[ A-B := \left\{ a-b : a \in A,\; b \in B \right\}.\footnote{For example, if $ A=\{0,1\} $ and $ B=\{1,2\} $ then $ A-B=\{-2,-1,0\} $ and $ A-A=\{-1,0,1\} $.} \]
	\begin{enumerate}
		\item Prove that if $ A $ is open then so is $ A-B $, for any $ B $.
            \begin{proof}
                Let $z$ be a point in $A - B$. By definition, there exist elements $a \in A$ and $b \in B$ such that $z = a - b$.

                Since $A$ is open and $a \in A$, $a$ is an interior point of $A$. Therefore, there exists an $\epsilon > 0$ such that the $r$-neighborhood $N_\epsilon(a) \subseteq A$.

                We aim to show there exists a $\delta > 0$ such that $N_\delta(z) \subseteq A - B$. Consider $\delta = \epsilon$ and let $z' \in N_\delta(z)$. This means $|z' - z| < \delta$.

                For $z' = a' - b$, we want to find an $a'$ such that $a' \in A$ (which implies $z' \in A - B$). Since $z' - z = (a' - b) - (a - b) = a' - a$, we have $|a' - a| < \delta = \epsilon$. This implies that $a' \in N_\epsilon(a)$. $A$ is open, meaning $N_\epsilon(a) \subseteq A$, so we have $a' \in A$. This means for every $z' \in N_\delta(z)$, we can find such an $a'$ in $A$. Therefore, $z' = a' - b \in A - B$, showing that $N_\delta(z) \subseteq A - B$.

                Since we have found a $\delta$-neighborhood of $z$ that is entirely contained within $A - B$, $z$ is an interior point of $A - B$. Because our choice of $z$ in $A - B$, every point in $A - B$ is an interior point of $A - B$. Therefore, $A - B$ is open.
            \end{proof}

		\item Give an example of a closed set $ A $ and a set $ B $ for which $ A-B $ is not closed. 
            \begin{proof}
                Let \(A = \{0\}\) and \(B = (0, 1)\). Then \[A - B = \{0 - b \mid b \in B\} = \{-b \mid b \in (0, 1)\} = (-1, 0)\] Since \((-1, 0)\) is not closed, we have found an example of a closed set \(A\) and a set \(B\) for which \(A - B\) is not closed.
            \end{proof}

            \item Prove that if the set $ A $ has a limit point then $ 0 $ is a limit point of $ A-A $.
            \begin{proof}
                Assume $A$ has a limit point, say $c$. By definition, for every $\epsilon > 0$, there exists a point $a \in A$ such that $a \neq c$ and $|a - c| < \epsilon$. This means that within any arbitrarily small distance around $c$, we can find a distinct point $a$ from $A$.

To show $0$ is a limit point of $A-A$, we must show that for every $\epsilon > 0$, there exists a $y \in A-A$ such that $y \neq 0$ and $|y - 0| < \epsilon$.

Since $c$ is a limit point of $A$, for any given $\epsilon > 0$, there exists $a_1, a_2 \in A$ such that $|a_1 - c| < \frac{\epsilon}{2}$ and $|a_2 - c| < \frac{\epsilon}{2}$, with $a_1 \neq a_2$ ($c$ being a limit point guarantees the presence of infinitely many points of $A$ in every neighborhood of $c$).

Consider $y = a_1 - a_2$, which is an element of $A-A$. We want to show that $|y| = |a_1 - a_2| < \epsilon$. Since both $a_1$ and $a_2$ are within $\frac{\epsilon}{2}$ of $c$, we can use the triangle inequality to deduce that:

$$|y| = |a_1 - a_2| = |(a_1 - c) - (a_2 - c)| \leq |a_1 - c| + |c - a_2| < \frac{\epsilon}{2} + \frac{\epsilon}{2} = \epsilon$$


This shows that $y = a_1 - a_2$ is within $\epsilon$ of $0$ and $y \neq 0$ since $a_1 \neq a_2$. Thus, for every $\epsilon > 0$, there exists a $y \in A-A$ such that $|y| < \epsilon$ and $y \neq 0$, proving that $0$ is a limit point of $A-A$.
            \end{proof}

	\end{enumerate}
\end{problem}
\medskip
%\vfill\hfill \emph{turn page}$ \longrightarrow $
%\newpage

\newpage
\begin{problem}{5}
	Provide an example of a subset $ E $ in $ \R^2 $ satisfying that 
	\begin{itemize}
		\item $ E $ has no interior points;
		\item $ E $ is not closed; but
		\item $ E \cup \{0\} $ is closed.
	\end{itemize}
	You do not need to provide a rigorous proof, but do explain your answers (an illustration may help!).

    \begin{proof}
        Let $E$ be the set of points in the plane that lie on the $x$-axis, excluding the origin. That is,
        $$E = \{(x, 0) \in \R^2 \, \vert \, x \neq 0\}$$

        \textbf{$E$ has no interior points:}
        An interior point of a set $E$ in $\mathbb{R}^2$ requires that there exists some $\epsilon > 0$ such that an $\epsilon$-neighborhood around that point is entirely contained within $E$. For any point $(x, 0) \in E$, every $\epsilon$-neighborhood will include points that do not lie on the $x$-axis (since neighborhoods in $\mathbb{R}^2$ are circular), thus including points not in $E$. Therefore, no point of $E$ is an interior point, as you cannot find an $\epsilon$-neighborhood around any point in $E$ that lies completely within $E$.

        \textbf{$E$ is not closed:}
        A set is closed if it contains all its limit points. The origin $(0,0)$ is a limit point of $E$ because, for any $\epsilon > 0$, the $\epsilon$-neighborhood of $(0,0)$ includes points of $E$ (any point $(x,0)$ with $0 < |x| < \epsilon$). Since $(0,0)$ is not included in $E$, $E$ does not contain all its limit points, making it not closed.

        \textbf{$E \cup \{0\}$ is closed:}
        By adding the origin $(0,0)$ to $E$, we form the set $E \cup \{0\} = {(x, 0) \in \mathbb{R}^2}$, which is the entire $x$-axis. This set includes all its limit points, including the origin. Any limit point of this set would have to lie on the $x$-axis, and since the set now includes every point on the $x$-axis, it is closed. There are no points in $\mathbb{R}^2$ that are limit points of $E \cup {0}$ without being included in the set itself.

    \end{proof}
\end{problem}


\vspace{1cm}\hfill \emph{Good luck!}

\end{document}
