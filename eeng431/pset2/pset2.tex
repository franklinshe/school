\documentclass[10pt]{article}
 
\usepackage[margin=1in]{geometry} 
\usepackage{amsmath,amsthm,amssymb, graphicx, multicol, array, bbm}
\usepackage{listings}
\usepackage{color}

\lstdefinestyle{Rstyle}{
  language=R,
  basicstyle=\small\ttfamily,
  keywordstyle=\color{blue},
  commentstyle=\color{black},
  stringstyle=\color{purple},
  numbers=left,
  numberstyle=\tiny,
  stepnumber=1,
  numbersep=5pt,
  backgroundcolor=\color{white},
  frame=single,
  rulecolor=\color{black},
  captionpos=b,
  breaklines=true,
  breakatwhitespace=true,
  title=\lstname,
  showstringspaces=false,
  tabsize=2,
  belowskip=1em,
  aboveskip=1em,
}

 
\newcommand{\N}{\mathbb{N}}
\newcommand{\Z}{\mathbb{Z}}
\newcommand{\R}{\mathbb{R}}
\newcommand{\C}{\mathbb{C}}
\newcommand{\A}{\mathcal{A}}
\newcommand{\E}{\mathbb{E}}
 
\newenvironment{problem}[2][Problem]{\begin{trivlist}
\item[\hskip \labelsep {\bfseries #1}\hskip \labelsep {\bfseries #2.}]}{\end{trivlist}}

\begin{document}


\title{\vspace{-2cm} EENG 431 - Homework 2}
\author{Franklin She}

\maketitle

\begin{problem}{2}
    \hfill
    \begin{enumerate}
        \item
            $$\mathcal{H}_{=k}^\mathcal{X} = \{h \in \{0, 1\}^\mathcal{H}: |\{x : h(x) = 1\}| = k\}$$
            First, we can see that $\text{VCdim}(\mathcal{H}_{=k}) \leq k$, because it is impossible for $h \in \mathcal{H}_{=k}$ to correctly label $k+1$ positive points. Secondly, we can see that $\text{VCdim}(\mathcal{H}_{=k}) \leq |\mathcal{X}| - k$, because it is impossible for $h \in \mathcal{H}_{=k}$ to correctly label $|\mathcal{X}| - k + 1$ negative points. Combining the two, $\text{VCdim}(\mathcal{H}_{=k}) \leq \min{\{k, |\mathcal{X}| - k + 1\}}$.
            
            Now, consider a set $C$ of size $\min{\{k, |\mathcal{X}| - k + 1\}}$. The total of positive points in $C$ is at most $k$. We can define $h$ as
            $$h(x) = \begin{cases}
                1 & \text{if } x \in \mathcal{X} \setminus C \\
                1 & \text{if } x \in C \text{ and has positive label} \\
                0 & \text{if } x \in C \text{ and has negative label} \\
                0 & \text{otherwise}
            \end{cases}$$
            This $h$ shatters this set $C$, therefore we can conclude that $\text{VCdim}(\mathcal{H}_{=k}) = \min{\{k, |\mathcal{X}| - k + 1\}}$.
        \item
            $$\mathcal{H}_{at-most-k} = \{h \in \{0, 1\}^\mathcal{H}: |\{x : h(x) = 1\}| \leq k
                \text{ or } |\{x : h(x) = 0\}| \leq k\}$$
            First, we can see that $\text{VCdim}(\mathcal{H}_{at-most-k}) \leq k$, because it is impossible for $h \in \mathcal{H}_{at-most-k}$ to correctly label $k+1$ positive points. 
            Now, consider a set $C$ of size $k$. We define $h$ as
            $$h(x) = \begin{cases}
                1 & \text{if } x \in C \text{ and has positive label} \\
                0 & \text{if } x \in C \text{ and has negative label} \\
                0 & \text{otherwise}
            \end{cases}$$
            This $h$ shatters this set $C$, therefore we can conclude that $\text{VCdim}(\mathcal{H}_{at-most-k}) = k$.
    \end{enumerate}
\end{problem}

\begin{problem}{5}
    $$\mathcal{H}_{\text{rec}}^d \text{ is the class of axis-aligned rectangles in } \R^d$$

    First, we can see that $\text{VCdim}(\mathcal{H}_{\text{rec}}^d) \leq 2d$, because it is impossible for $h \in \mathcal{H}_{\text{rec}}^d$ to shatter a set $C$ with $2d+1$ points. We locate an "inside" point $x \in C$, defined as $\forall j \in [d]$, $\exists x', x'' \in C$ such that $x'_j \leq x_j$ and $x''_j \geq x_j$. Setting the label of $x$ as negative, and all other points as positive is an example of a configuration in which all points cannot be correctly labeled by $h \in \mathcal{H}_{\text{rec}}^d$.
    
    Now considering a set $C$ of size $2d$, positioning each point in a way such that $\forall x, x' \in C$ and $\forall j \in [d]$, $x_j \neq x'_j$, we have a set in which $\mathcal{H}_{\text{rec}}^d$ shatters. Taking any arbitrary subset of $C$ to be positive labeled points and the rest negative, we can still find a $\R_d$ rectangle identifying only and all positive labeled points.

\end{problem}

\begin{problem}{7}
    \hfill
    \begin{enumerate}
        \item
            We have seen in class that the set of threshold functions:
            $$\mathcal{H} = \{h_a : a \in \R\} \text{ where } h_a(x) = \mathbbm{1}_{x \leq a}$$
            is infinite, but $\text{VCdim}(\mathcal{H}) = 1$.
        \item 
            Consider
            $$\mathcal{H} = \{h_1, h_{1/2}\}$$
            We have 
            $\lfloor \log_2(|\mathcal{H}|) \rfloor = \lfloor \log_2(2) \rfloor = 1$.
            We also have
            $\text{VCdim}(\mathcal{H}) = 1$
    \end{enumerate}
\end{problem}

\begin{problem}{9}
    $$\mathcal{H} = \{h_{a,b,s} : a \leq b, s \in \{-1, 1\}\}$$ where
    $$h_{a,b,s}(x) = \begin{cases}
        s & \text{if } x \in [a, b] \\
        -s & \text{if } x \notin [a, b]
    \end{cases}$$
    As this was an example in class, we will show $\text{VCdim}(\mathcal{H}) = 3$. We will approach this explicitly finding a set $C$ with three elements shattered by $\mathcal{H}$ and then showing that any set of four elements cannot be shattered by $\mathcal{H}$.
    Let $C = \{0, 1, 2\}$. For each of the $2^3 = 8$ configurations, we will define $h_{a,b,s}$ correctly labeling all elements.
    \begin{itemize}
        \item $((0, -1), (1, -1), (2, -1))$: $a = -0.5$, $b = 2.5$, $s = -1$
        \item $((0, -1), (1, -1), (2, +1))$: $a = 1.5$, $b = 2.5$, $s = +1$
        \item $((0, -1), (1, +1), (2, -1))$: $a = 0.5$, $b = 1.5$, $s = +1$
        \item $((0, -1), (1, +1), (2, +1))$: $a = 0.5$, $b = 2.5$, $s = +1$
        \item $((0, +1), (1, -1), (2, -1))$: $a =-0.5$, $b = 0.5$, $s = +1$
        \item $((0, +1), (1, -1), (2, +1))$: $a = 0.5$, $b = 1.5$, $s = -1$
        \item $((0, +1), (1, +1), (2, -1))$: $a = -0.5$, $b = 1.5$, $s = +1$
        \item $((0, +1), (1, +1), (2, +1))$: $a = -0.5$, $b = 2.5$, $s = +1$
    \end{itemize}
    Now consider a set $\{a, b, c, d\}$. WLOG, take $a < b < c < d$. The labeling $((a, +1), (b, -1), (c, +1), (d, -1))$ cannot be correctly labeled by any $h \in \mathcal{H}$.

\end{problem}

\begin{problem}{11 (part 1)}
    Let $d = \max_i \text{VCdim}(\mathcal{H}_i)$ and $d \geq 3$. Let $\text{VCdim}(\bigcup_i^r \mathcal{H}_i) = k$.
    Because the VC dimension of the union of hypothesis classes is at most any of the VC dimensions of the individual classes, we know that $\forall 1 \leq i \leq r$, $\text{VCdim}(\mathcal{H}_i) \leq d \leq k$.
    Let $C$ be a subset of size $k$ in which $\bigcup_i^r \mathcal{H}_i$ shatters, meaning $\bigcup_i^r \mathcal{H}_i$ can produce all $2^k$ labeling configurations of $C$.
    By Sauer's lemma, we have $\tau_{\mathcal{H}_i}(k) \leq \left(\frac{ek}{d}\right)^d$. As $d \geq 3$, we can bound this with $\tau_{\mathcal{H}_i}(k) \leq k^d$.
    By the definition of the growth function, we also have $\tau_\mathcal{H}(k) \leq \sum_{i = 1}^r \tau_{\mathcal{H}_i}(k)$.
    Therefore, we have $\tau_\mathcal{H}(k) \leq r k^d \implies 2^k \leq r k^d \implies k \leq d \log k + \log r$. Using Lemma A.2, we have $k < 4d \log (2d) + 2 \log r$.
\end{problem}

\end{document}
