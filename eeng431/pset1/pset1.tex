\documentclass[10pt]{article}
 
\usepackage[margin=1in]{geometry} 
\usepackage{amsmath,amsthm,amssymb, graphicx, multicol, array, bbm}
\usepackage{listings}
\usepackage{color}

\lstdefinestyle{Rstyle}{
  language=R,
  basicstyle=\small\ttfamily,
  keywordstyle=\color{blue},
  commentstyle=\color{black},
  stringstyle=\color{purple},
  numbers=left,
  numberstyle=\tiny,
  stepnumber=1,
  numbersep=5pt,
  backgroundcolor=\color{white},
  frame=single,
  rulecolor=\color{black},
  captionpos=b,
  breaklines=true,
  breakatwhitespace=true,
  title=\lstname,
  showstringspaces=false,
  tabsize=2,
  belowskip=1em,
  aboveskip=1em,
}

 
\newcommand{\N}{\mathbb{N}}
\newcommand{\Z}{\mathbb{Z}}
\newcommand{\R}{\mathbb{R}}
\newcommand{\C}{\mathbb{C}}
\newcommand{\A}{\mathcal{A}}
\newcommand{\E}{\mathbb{E}}
 
\newenvironment{problem}[2][Problem]{\begin{trivlist}
\item[\hskip \labelsep {\bfseries #1}\hskip \labelsep {\bfseries #2.}]}{\end{trivlist}}

\begin{document}


\title{\vspace{-2cm} EENG 431 - Homework 1}
\author{Franklin She}

\maketitle

\section{Chapter 2}
 
\begin{problem}{1}
    Let's recall the predictor in Equation (2.3),
    \begin{align*}
        h_S(x) = \begin{cases}
            y_i & \text{if } \exists i \in [m] \text{ s.t. } x_i = x\\
            0 & \text{otherwise}
        \end{cases}
    \end{align*}
    The goal is to construct $p_S(x)$ in such a way that it outputs a non-negative value if and only if $x$ matches one of the $\mathbf{x}_i$ where $f(\mathbf{x}_i)) = 1$ and to a negative value otherwise.

    Let's consider the case of only one positive example in the training set, or $S = \{(\mathbf{x}_1, 1)\}$. Then, we can define $p_S(x)$ as follows:
    \begin{align*}
        p_S(x) = -  \lVert x - \mathbf{x}_1 \rVert
    \end{align*}
    We know that $p_S(\mathbf{x}_1) = 0$ and $p_S(x) < 0$ for all $x \neq \mathbf{x}_1$.

    To generalize this to the case of multiple examples, we can define $p_S(x)$ as follows:
    \begin{align*}
        p_S(x) = -\prod_{i=1 \colon y_i = 1}^m \left(\lVert x - \mathbf{x}_i \rVert \right)
    \end{align*}
    Notice by taking the product of the distances, we ensure that if $x$ matches any of the $\mathbf{x}_i$ where $f(\mathbf{x}_i) = 1$, then $p_S(x) = 0$. Otherwise, $p_S(x) < 0$.
\end{problem}

\begin{problem}{2}
    \begin{align*}
        \E_{S \mid_x \sim D^m}[L_S(h)] &= \E_{S \mid_x \sim D^m} \left[ \frac{1}{m} \sum_{i=1}^m \mathbbm{1}(h(x_i) \neq y_i) \right] \\
                                       &= \frac{1}{m} \sum_{i=1}^m \E_{x\sim D} \left[ \mathbbm{1}(h(x_i) \neq y_i) \right] & \text{by linearity of expectation}\\
                                       &= \frac{1}{m} \sum_{i=1}^m \Pr_{x\sim D} \left[ h(x_i) \neq y_i \right] & \text{$x_1, \ldots, x_m$ are i.i.d} \\ 
                                       &= \frac{1}{m}\cdot m\cdot  L_{D,h}(h) \\
                                       &= L_{D,h}(h)
    \end{align*}
\end{problem}

\begin{problem}{3} \hfill
    \begin{enumerate}
        \item To show that $A$ is an ERM, we can show that $A$ returns $h^*$ s.t. $L_{D, f}(h^*) = 0$.
            By definition, $A$ labels correctly all the positive examples in the training set. Because we also assume realizability and $A$ is the tightest rectangle, all negative examples in the training set are also correctly labeled. Therefore, $A$ labels the whole training set correctly, so $A$ must be an ERM.
        \item 
            Let $\mathcal{D}$ be a distribution over $\mathcal{X}$.
            We will use $R^* = R(a_1^*, a_2^*, a_3^*, a_4^*)$ defined in the hint, and let $f$ be its corresponding hypothesis.
            Let $R(S)$ be the rectangle returned by the algorithm $A$ given the training set $S$.
            We can first notice that $R(S) \subseteq R^*$ because of the way $R$ is defined. Thus we have
            \begin{align*}
                L_{\mathcal{D}, f}(A(S)) & = \mathcal{D}(\{x \in \mathcal{X} : A(S)(x) \neq f(x)\}) \\
                                        &= \mathcal{D}(\{x \in \mathcal{X} : x \not\in S \text{ and } f(x) = 1\}) \\
                                        &= \mathcal{D}(R^* \setminus R(S))
            \end{align*}

            Next, we consider the rectangles $R_1$, $R_2$, $R_3$, and $R_4$ defined in the hint, which all have a probability mass of $\epsilon / 4$.
            We can deduce $L_{\mathcal{D}, f}(A(S)) \leq \epsilon$ if $S$ contains positive examples in all the rectangles
            $R_1$, $R_2$, $R_3$, and $R_4$. The probability that $S$ contains no positive examples in any of the rectangles is at most $(1 - \epsilon / 4)^m \leq e^{(-\epsilon/4)m}$. By the union bound, we have
            \begin{align*}
                \mathcal{D}(\{S : S \cap R_i = \emptyset \text{ for some } i \in \{1, 2, 3, 4\}\}) & \leq 4 \cdot e^{(-\epsilon/4)m}
            \end{align*}
            Plugging in the value of $m$ of $\geq \frac{4 \log(4/\delta)}{\epsilon}$, we can see that $A$ will return a hypothesis with error of at most $\epsilon$ with probability at least $1 - \delta$.
    \end{enumerate}
\end{problem}

\section{Chapter 3}

\begin{problem}{2}
    \hfill
    \begin{enumerate}
        \item 
        If a positive $x$ appears in $S$, we can return the true $h_x$. Otherwise, we can return the all-negative hypothesis.
        \begin{align*}
            h_S(x) = \begin{cases}
                h_x & \text{if } \exists x \in S \text{ s.t. } f(x) = 1\\
                h^- & \text{otherwise}
            \end{cases}
        \end{align*}
        Because we assume realizability, we have $L_S(h_S) = 0$, so the algorithm that returns the hypothesis $h_S$ is an ERM.
        \item 
            Let $\mathcal{D}$ be a distribution over $\mathcal{X}$.
            First we notice that if the true hypothesis is $h^-$, our algorithm returns the perfect hypothesis.

            Now assume that there exists a positive example $x$ such that $f(x) = 1$. Because of the realizaibility assumption, $x$ is unique. If this $x$ is in our sample, then our algorithm returns the perfect hypothesis again. Also note that if $\mathcal{D}(x) \leq \epsilon$, then $L_{\mathcal{D}}(h) \leq \epsilon$ with probability 1.

            To find an upper bound on sample complexity, we are thus interested in the event where $x$ does not appear in our sample, and $\mathcal{D}(x) > \epsilon$.
            This means that $\mathcal{D}(x') \leq 1 - \epsilon$ for all $x' \neq x$.
            Therefore, sampling $m$, we have
            \begin{align*}
                \mathcal{D}^m(\{S : L_{\mathcal{D}}(h_S) > \epsilon\}) \leq (1 - \epsilon)^m \leq e^{-\epsilon m}
            \end{align*}
            Picking $\delta$ such that $e^{-\epsilon m} \leq \delta$, we can solve for $m$ to show that 
            \begin{align*}
                m_\mathcal{H}(\epsilon, \delta) \leq \left\lceil\frac{\log(1/\delta)}{\epsilon}\right\rceil
            \end{align*}


    \end{enumerate}
\end{problem}

\begin{problem}{3}

Let \(A\), given a training set \(S\) produce a hypothesis \(h_{\hat{r}}\) corresponding to the smallest circle that encloses all positive examples in \(S\), where \(\hat{r}\) denotes the radius of this circle.

Define \(C^*\) as the circle corresponding to \(h^*\) (realizability assumption hypothesis) with radius \(r^*\), and \(C(\hat{r})\) as the circle produced by \(A\) with radius \(\hat{r}\). We first notice that \(C(\hat{r}) \subseteq C^*\).

Let \(r_1 \leq r^*\) be such that the circular strip \(E = \{x \in \mathbb{R}^2 : r_1 \leq \|x\| \leq r^*\}\) has a probability mass. This implies that the probability that a randomly drawn sample falls within \(E\) is exactly \(\epsilon\).

If the training set \(S\) contains at least one positive example from \(E\), the hypothesis \(h_{\hat{r}}\) produced by \(A\) will have a generalization error of at most \(\epsilon\).

The probability that no sample in \(S\) falls within \(E\) is at most \((1 - \epsilon)^m\), as each sample independently has a probability of \(\epsilon\) to fall within \(E\).
            \begin{align*}
                \mathcal{D}^m(\{S : L_{\mathcal{D}}(h_S) > \epsilon\}) \leq (1 - \epsilon)^m \leq e^{-\epsilon m}
            \end{align*}
            Picking $\delta$ such that $e^{-\epsilon m} \leq \delta$, we can solve for $m$ to show that 
            \begin{align*}
                m_\mathcal{H}(\epsilon, \delta) \leq \left\lceil\frac{\log(1/\delta)}{\epsilon}\right\rceil
            \end{align*}

\end{problem}

\end{document}
